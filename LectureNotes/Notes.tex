\documentclass[11pt]{article}
\usepackage{amsmath}
\usepackage{enumerate}
\usepackage{amsfonts}
\usepackage{graphicx}
\usepackage{amsthm}

\theoremstyle{plain}
\newtheorem{theorem}{Theorem}[section]
\newtheorem{corollary}[theorem]{Corollary}
\newtheorem{lemma}[theorem]{Lemma}
\newtheorem{proposition}[theorem]{Proposition}

\theoremstyle{definition}
\newtheorem{definition}[theorem]{Definition}
\newtheorem{remark}{Remark}
\newtheorem{example}[theorem]{Example}

\newcommand{\re}[0]{\text{Re}}
\newcommand{\im}[0]{\text{Im}}
\newcommand{\C}{\mathbb{C}}
\newcommand{\Ind}{\text{Ind}}
\newcommand{\Z}{\mathbb{Z}}
\newcommand{\N}{\mathbb{N}}
\newcommand{\Res}{\text{Res}}
\newcommand{\sla}{\char`\\}

\title{Math H185 Lecture Notes}
\author{Eric Xia}
\begin{document}

\maketitle

\section{Preliminaries}

Here is an important property of complex numbers:
$$ |\text{Re}(z)| \leq |z|, |\text{Im}(z)| \leq |z|, |z| \leq |\text{Re}(z)| + |\text{Im}(z)| $$

\begin{lemma}
Let $(a_n)$ be a sequence in $\mathbb{C}$, and $L \in \mathbb{C}$. Then $\lim_{n \to \infty} a_n = L$ if and only if 
$$ \lim_{n \to \infty} \text{Re}(a_n) = \text{Re}(L), \hspace{2mm} \lim_{n \to \infty} \text{Im}(a_n) = \text{Im}(L) $$
\end{lemma}

\begin{lemma}
Let $F \subset \mathbb{C}$ be a set. Then the following are equivalent
\begin{enumerate}[(i)]
\item $F$ is closed
\item for every sequence $(z_n) \in F$, and $z \in \mathbb{C}$, with $\lim_{n \to \infty} z_n = z$, it follows that $z \in F$
\end{enumerate}
\end{lemma}

Proof: This is the definition of a closed set.
\newline

\noindent
Definition: A \textbf{Cauchy sequence} $z_n$ is a sequence in which for every $\epsilon > 0$ there exists $N \in \mathbb{N}$ such that for every pair $m, n \geq N$, we have that $d(z_n, z_m) < \epsilon$.
\newline

\noindent
Definition: A set $S$ is \textbf{complete} if every Cauchy sequence in $S$ converges to some value in $S$.

\begin{theorem}
$\mathbb{C}$ is complete.
\end{theorem}

Proof: We will use the property that $\mathbb{R}$ is complete. Suppose we have a Cauchy sequence $(z_n)$ in $\mathbb{C}$. Thus given $\epsilon > 0$ we have $N$ as defined above. Thus, $\forall m,n \geq N$, $|z_m - z_n| < \epsilon$. Then we have that 
$$ |\text{Re}(z_m - z_n)| \leq |z_m - z_n| < \epsilon, \hspace{2mm} |\text{Im}(z_m - z_n)| \leq |z_m - z_n| < \epsilon $$
Note that $\text{Re}(z_m - z_n) = \text{Re}(z_m) - \text{Re}(z_n)$ and the same for the imaginary component. Thus the real components and imaginary components of $z_n$ are Cauchy and thus convergent, so $(z_n)$ is convergent. 

\begin{definition}
A set $K \in \mathbb{C}$ is called \textbf{sequentially compact} if every sequence in $K$ has a convergent subsequence which converges to a point in $K$.
\end{definition}

\begin{proposition}
If $K \in \mathbb{C}$, then $K$ is sequentially compact if and only if $K$ is closed and bounded.
\end{proposition}

Proof: This proof is made trivial if you use the fact that sequential compactness is equivalent to covering compactness. Suppose $K$ is sequentially compact. Then let $(x_n)$ be a sequence in $K$ that converges to some $L$ in $\mathbb{C}$. By compactness, there exists a subsequence of $(x_{n_k})$ that is convergent to some value in $K$. Since in a convergent sequence every subsequence converges to the same value, $L \in K$ and so $K$ is closed. 

To verify boundedness, suppose $K$ is not bounded. This implies that $\forall x \in K$, and $r \in \mathbb{R}$, there exists $y \in K$ such that $|x - y| > r$ (otherwise $K$ would be bounded). Let $r \in \mathbb{R}^{+}$ and construct the sequence inductively as follows. Let  $x_1 \in K$. Assuming $x_1, ..., x_n$ is defined, pick $x_{n+1}$ such that $d(x_{n+1}, x_{i}) > r$. This must exist, otherwise $K \subset \cup_{i = 1}^{n} B_{r}(x_i)$ where $B_r(x_i)$ is the ball centered around $x_i$ with radius $r$ implying that $K$ is bounded. Then in our sequence $(x_n)$ we have that distance between every pair of points is at least $r$, which is a property that carries over to every subsequence. Therefore every subsequence does not converge and so $K$ is not sequentially compact, proving the contrapositive.

The other way is a bit more complicated and I don't really want to type it up, so it is left as an exercise to the reader.

\begin{proposition}
Let $K_i$ be a sequence of nonempty sequentially compact subsets of $\mathbb{C}$, and suppose we have that $K_{i+1} \subset K_{i}$ for every $i$. Then $\cap_{n=1}^{\infty}K_n$ is nonempty.
\end{proposition}

Proof: For each $K_i$, pick element $x_i \in K_i$. We have that the sequence $(x_n) \in K_1$, so it contains a convergent subsequence, $(x_{n_k})$ in $K_1$. Let $x$ be the value of which it converges to. $x$ is clearly in $K_i$ for every $i$, and thus $x \in \cap_{n=1}^{\infty}K_n$.

\begin{proposition}
Let $D \in \mathbb{C}$, and $f: D \to \mathbb{C}$ a function with $z_0 \in D$. Then the following are equivalent:
\begin{enumerate}[(i)]
\item  For every sequence $(d_n) \in D$ with $\lim_{n \to \infty}d_n = z_0$, it follows that $\lim_{n\to\infty}f(d_n) = f(z_0)$
\item $\forall\epsilon > 0 $ $\exists \delta > 0$ such that $\forall d \in D$ with $|d - z_0| < \delta$ it follows that $|f(d) - f(z_0)| < \epsilon$
\item For every open set $V \in \mathbb{C}$ with $f(z_0) \in V$, there exists open $U \in \mathbb{C}$ with $z_0 \in U$ and $U \cap D \subset f^{-1}(V)$
\end{enumerate}
\end{proposition}

Proof: Suppose that (ii) is false, that $\exists \epsilon > 0$ such that $\forall \delta$, $\exists d \in D$ with $|d - z_0| > \delta$ such that $|f(d) - f(z_0)| > \epsilon$. Then construct sequence $(d_n)$ such that $|d_n - z_0| > \frac{1}{n}$ and $|f(d_n) - f(z_0)| > \epsilon$. We have that $\lim_{n \to \infty}d_n = z_0$, but since for all $n$, $|f(d_n) - f(z_0)| > \epsilon$, $f(d_n)$ does not converge to $z_0$. Thus (i) implies (ii) by contraposition, as desired.

Now suppose that (ii) is true. By definition of open, $\exists \epsilon > 0$ such that $|f(z_0) - v| < \epsilon$ implies that $v \in V$. By (ii), $\exists \delta > 0$ such that $d \in B_{\delta}(z_0) \cap D$ (which implies that $d \in D$ and $|d - z_0| < \delta$) implies $|f(d) - f(z_0)| < \epsilon$, and so $d \in f^{-1}(V)$ and we conclude $B_{\delta}(z_0) \cap D \subset f^{-1}(V)$

Finally, suppose (iii) is true. Consider a sequence $(d_n) \to z_0$. Take $\epsilon > 0$ and consider $B_{\epsilon}(f(z_0))$. It is open so there exists open $D$ with $z_0 \in D$ and $U \cap D \subset f^{-1}(B_{\epsilon}(f(z_0)))$. Thus there exists $\delta > 0$ such that $B_{\delta}(z_0) \in D$, and by definition of convergence $\exists N$ such that $\forall n \geq N$ we have $|d_n - z_0| < \delta$. Then we have that since $d_n \in U \cap D$, we have that $f(d_n) \in B_{\epsilon}(f(z_0))$ so $|f(d_n) - f(z_0)| < \epsilon$ for all $n \geq N$, and so 
$$ \lim_{n\to\infty} f(d_n) = f(z_0) $$

\begin{definition}
A function $f$ is \textbf{continuous} if it satisfies one of the three conditions stipulated above.
\end{definition}

\begin{remark}
If $f: D \to \mathbb{C}$ and $E \subset D$, define $g = f_{|E}$ (ie $f$ restricted to $E$), then $f$ being continuous implies that $g$ is continuous. 
\end{remark}

\begin{lemma}
Let $K \in \mathbb{C}$ be sequentially compact and $f: K \to \mathbb{C}$ continuous, then $f(K)$ is sequentially compact.
\end{lemma}

Proof: Let $(a_n)$ be a sequence in $f(K)$. Define $d_n = f^{-1}(a_n)$ and if there are multiple, arbitrarily pick one. We have by sequential compactness of $K$, some subsequence $d_{n_{k}}$ converges to a $d \in K$. Thus by continuity, $f(d_{n_k}) \to f(d)$ as $k \to \infty$, and thus $(a_{n_k})$ converges.

\subsection{Differentiability}

\begin{definition}
Let $D \subset \mathbb{C}$, and $z_0 \in \mathbb{C}$, then $z_0$ is called a \textbf{cluster point} in $D$ if there exists a sequence $(d_n) \in D - \{z_0\}$ such that $\lim_{n\to\infty} d_n = z_0$.
\end{definition}

\begin{definition}
Let $D \subset \mathbb{C}$, $f: D \to \mathbb{C}$, and $z_0 \in \mathbb{C}$, and $z_0$ is a cluster point of $D$. Let $L \in \mathbb{C}$, then we say $f$ has a limit $L$ at $z_0$ if $\forall \epsilon > 0$, $\exists \delta > 0$ such that $\forall d \in D$ such that $0 < |d - z_0| < \delta$, it follows that $|f(d) - L| < \epsilon$.
\end{definition}

\begin{lemma}
$D \subset \mathbb{C}$, $z_0$ is a cluster of $D$, $f: D \to \mathbb{C}$ and $L \in \mathbb{C}$. Define $g: D \cup \{z_0\} \to \mathbb{C}$ as 
$$ g(z) = \begin{cases}
f(z) & \text{if }z \in D - \{z_0\} \\
L & \text{if } z = z_0
\end{cases} $$
Then $f$ has a limit $L$ at $z_0$ if and only if $g$ is continuous at $z_0$.
\end{lemma}

Proof: This holds from the definition of continuity and Definition 1.11.

\begin{definition}
Let $D \subset \mathbb{C}$, $f: D \to \mathbb{C}$, $z_0$ a cluster of $D$. Then $f$ is differentiable at $z_0$ if $\exists L \in \mathbb{C}$ such that 
$$ z \mapsto \begin{cases}
\frac{f(z) - f(z_0)}{z - z_0} & \text{if } z \neq z_0 \\
L & \text{if } z = z_0
\end{cases} $$
is continuous at $z_0$. We then write $f'(z_0) = L$.
\end{definition}

\begin{remark}
$f$ is differentiable at $z$ if and only if 
$$ \lim_{z \to z_0} \frac{f(z) - f(z_0)}{z - z_0}$$
exists.
\end{remark}

\begin{proposition}
Let $D \subset \mathbb{C}$, $f: D \to \mathbb{C}$, $z$ a cluster of $D$, $L \in \mathbb{C}$, then the following are equivalent:
\begin{enumerate}[(i)]
\item $f$ is differentiable at $z_0$ with $f'(z_0) = L$
\item There exists function $ \varphi : D \to \mathbb{C}$ continuous at $z_0$ such that $\varphi(z_0) = L$ and 
$$ f(z) = f(z_0) + (z - z_0) \varphi(z) \hspace{2mm} \forall z \in D $$
\end{enumerate}
\end{proposition}

Proof: We will show that (i) implies (ii), and the remaining are trivial. Define 
$$ \varphi(z) = \begin{cases}
\frac{f(z) - f(z_0)}{z - z_0} & \text{if } z \neq z_0 \\
L & \text{if } z = z_0
\end{cases} $$
Then we have that 
$$ f(z) = f(z_0) + (z - z_0)\varphi(z)$$ 
and that $\varphi(z)$ is continuous at $z_0$.

\begin{proposition}
Differentiability implies continuity.
\end{proposition}

Proof: Given $\epsilon > 0$, there exists $\delta >0$ such that 
$$ 0 < |z - z_0| < \delta \implies \left|\frac{f(z) - f(z_0)}{z - z_0}\right| < \epsilon $$
Thus 
$$ |f(z) - f(z_0)| < \epsilon |z - z_0| < \epsilon \delta < \epsilon $$
for sufficiently small $\delta$.

\begin{proposition}
Let $D \subset \mathbb{C}$, $z_0$ cluster of $D$, $f, g: D \to \mathbb{C}$ be differentiable at $z_0$. Then $f+g$, $\lambda f$, and $fg$ are differentiable. If $f(z_0) \neq 0$, then $\frac{1}{f}$ is differentiable.
\end{proposition}

Proof: Follows from the definition of differentiability. Only the proof for $fg$ is a little more involved. 

\begin{theorem}
Let $D_1, D_2 \subset \mathbb{C}$, $z_i$ a cluster for $D_i$, $f: D_{1} \to \mathbb{C}$ differentiable at $z_1$, and $g: D_2 \to \mathbb{C}$ differentiable at $z_2$ with $f(z_1) = z_2$. Suppose $f(D_1) \subset D_2$. Then $g \circ f$ is differentiable at $z_1$ and 
$$ (g \circ f)'(z_1) = g'(f(z_1)) f'(z_1) $$
\end{theorem}

Proof: Proposition 1.14 gives functions $\varphi_1$, $\varphi_2$ continuous at $z_1$, $z_2$ respectively, such that 
$$ \varphi_1(z_1) = f'(z_1), \hspace{2mm} \varphi_2(z_2) = g'(z_2) $$
and we have that 
$$ (g \circ f)(z) = g(z_2) + (f(z) - z_2)\varphi_2(f(z)) $$
$$ = (g \circ f)(z_1) + (z - z_1)\varphi_1(z)\varphi_2(f(z)) = (g \circ f)(z_1) + (z - z_1)\varphi(z)$$
with $\varphi: D_1 \to \mathbb{C}$ continuous at $z_1$ and defined as $\varphi(z) = \varphi_1(z)\varphi_2(f(z))$ proving the desired.
\newline

\begin{definition}
A \textbf{holomorphic} function is a differentiable function $f: U \to \mathbb{C}$ hwere $U$ is an open subset of $\mathbb{C}$.
\end{definition}

\vspace{5mm}
\noindent
Define $\Phi: \mathbb{R}^{2} \to \mathbb{C}$ as $(x, y) \mapsto x + iy$ and $\Psi : \mathbb{C} \to \mathbb{R}^2$ as the inverse. Let $U \subset \mathbb{C}$, and define $\tilde{U} = \Psi(U)$. If $f: U \to \mathbb{C}$, we can define $\tilde{f} = \Psi \circ f \circ \Phi$. Note $\tilde{f} : \tilde{U} \to \mathbb{R}^{2}$.

\vspace{5mm}
\noindent
Let $z_0 = z_0 + iy_0$. By definition, $\tilde{f}$ is differentiable at $(x_0, y_0)$ if and only if there exists $2 \times 2$ matrix $M$ and a function $r: \tilde{U} \to \mathbb{R}^2$ such that for every $(x, y) \in \tilde{U}$ we have 
$$ \tilde{f}(x, y) = \tilde{f}(x_0, y_0) + M\begin{bmatrix} x - x_0 & y - y_0 \end{bmatrix}^{\top} + r(x, y) $$
where 
$$ \lim_{(x, y) \to (x_0, y_0)} \frac{r(x, y)}{\sqrt{(x - x_0)^2 + (y - y_0)^2}} = 0$$
We say that $M$ is the derivative of $\tilde{f}$ at $(x_0, y_0)$.

\vspace{5mm}
\noindent
If $a, b \in \mathbb{R}$ with $L = a+ ib$, then we have that 
$$ \Psi((z - z_0)L) = \begin{bmatrix} a & -b \\ b & a \end{bmatrix} \begin{bmatrix} x- x_0 \\ y - y_0 \end{bmatrix} $$

\vspace{5mm}
\noindent
Suppose $f$ is differentiable at $z_0$, and $M = \begin{bmatrix} a & -b \\ b & a \end{bmatrix}$. Then we have that $\tilde{f} = \Psi \circ f \circ \Phi$ is differentiable at $(x_0, y_0)$ with derivative $M$ and 
$$r(x, y) = \Psi\left((x + iy) - z_0) * \psi(x + iy)\right)$$
where the $\psi$ comes from Proposition 1.14(III).

\vspace{5mm}
\noindent
Suppose $\tilde{f}: \mathbb{R}^{2} \to \mathbb{R}^{2}$ is differentiable at $(x_0, y_0)$ with derivative \\$M = \begin{bmatrix} a & -b \\ b & a \end{bmatrix}$, where $a, b \in \mathbb{R}$. Let 
$$ \psi(x + iy) = \begin{cases} 
\frac{\Phi(r(x, y))}{x + iy - z} & x +iy \neq z_0 \\
0 & x + iy = z_0 
\end{cases} 
$$
These satisfies Proposition 1.14(III) with $L = a + bi$, so $f$ is complex differentiable.

\vspace{5mm}
\noindent
Take $f: U \to \mathbb{C}$ continuous. Define $u, v: \tilde{U} \to \mathbb{R}$ via 
$$\tilde{f}(x,y) = (u(x, y), v(x, y)) $$

\begin{proposition}
The following are equivalent
\begin{enumerate}[(i)]
\item $f$ is differentiable at $z_0$
\item the functions $u,v$ are differentiable at $(x_0, y_0)$ and 
$$ (D_1 u)(x_0, y_0) = (D_2 v)(x_0, y_0) $$
$$ (D_2 u)(x_0, y_0) = -(D_1 v)(x_0, y_0) $$
\end{enumerate}
where $D_1 u = \delta_x u$, $D_2 u = \delta_y u$.
\end{proposition}

\begin{definition}
The equations are called the \textbf{Cauchy-Riemann equations}.
\end{definition}

\subsection{Series in $\mathbb{C}$}
Let $a_0, a_1, ... \in \mathbb{C}$, $\forall n \in N_0$. Define $s_n = \sum_{k=0}^{n} a_n \in \mathbb{C}$. We call $s_n$ the $n^{th}$ partial sum. We say that $\sum_{a_k}$ is convergent if and only if $(s_n)$ converges, and we write
$$ \sum_{k=0}^{\infty} a_k = \lim_{n \to \infty}s_n $$

\begin{remark}
$\sum a_n$ converges if and only if $\sum \text{Re}(a_n)$ and $\sum\text{Im}(a_n)$ converges.
\end{remark}

We say that the series $\sum a_k$ is absolutely convergent if $\sum |a_n|$ is convergent in $\mathbb{R}$.

\begin{theorem}
Every absolutely convergent sequence is convergent in $\mathbb{C}$.
\end{theorem}

Proof: 
$$ \sum_{n = 0}^{k} a_n \leq \left| \sum_{n=0}^{k} a_n\right| \leq \sum_{n = 0}^{k} |a_n|$$
and then by the sequence comparison test we have $\sum a_n$ converges (I think).

\begin{definition}
Let $D \subset \mathbb{C}$ and $f, f_0, f_1, ... : D \to \mathbb{C}$ be a sequence of functions. We say $\lim_{n \to \infty} f_n = f$ uniformly if $\forall \epsilon > 0$ there exists $N \in \mathbb{N}$ such that $\forall x \in D$, $\forall n \geq N$ we have $|f_n(x) - f(x)| < \epsilon $
\end{definition}

\begin{remark}
Consider $\sum f_k$ where $f_k : D \to \mathbb{C}$. Define $S_k = \sum_{i = 0}^{k} f_i$. The series $\sum f_{k}$ is uniformly convergent on $D$ if the sequence of partial sums $S_k$ is uniformly convergent.
\end{remark}

The following result is called the Weierstrass M-test.

\begin{proposition}
Let $D \subset \mathbb{C}$ and $f_0, f_1, ... : D \to \mathbb{C}$ be a sequence of functions. Let $a_0, a_1, ... \in \mathbb{R}^{+}$. Suppose $|f_k(z)| \leq a_k$ for all $z\in D$ and all $k \in \mathbb{N}_{0}$. Moreover suppose that the series $\sum a_n$ is convergent. Then the series $\sum f_n$ is uniformly convergent on $D$.
\end{proposition}

Proof: $\forall z \in D$, the series $\sum_{n} f_n(z)$ is absolutely convergent clearly (and so therefore convergent). Define $S: D \to \mathbb{C}$ by $S(z) = \sum_{n = 0}^{\infty} f(z)$. Let $z \in D$ and if $N \geq n$, then 
$$ \left| \sum_{k=0}^{N}f_k(z)- \sum_{k = 0}^{n}f_k(z)\right| = \left| \sum_{k = n+1}^{N} f_k(z) \right| \leq \sum_{k = n+1}^{N} |f_k(z)| \leq \sum_{k = n+1}^N a_k \leq \sum_{k = n+1}^{\infty} a_k $$
Now take the limit as $N \to \infty$, we get that 
$$ \left|S(z) - \sum_{k = 0}^n f_k(z)\right| \leq \sum_{k = n+1}^{\infty} a_k $$
Thus since $\lim_{n \to\infty} \sum_{k = n+1}^{\infty} a_k = 0$, we have the desired result.

\subsection{Integration}
Let $a, b \in \mathbb{R}$ with $a<b$ and $f:[a,b] \to \mathbb{C}$. This is called Riemann integrable over $[a, b]$ if both $\re f$ and $\im f$ are Riemann integrable on $[a,b] \to \mathbb{R}$, and 
$$ \int_a^b f(t)dt = \int_a^b \re f(t)dt + i\int_a^b \im f(t)dt $$

\begin{lemma}
The integral is linear.
\end{lemma}

\begin{lemma}
Let $a, b \in \mathbb{R}$, with $a<b$, and $f:[a,b] \to \mathbb{R}$ continuous. Then $|f|$ is Riemann integrable over $[a,b]$ and 
$$ \left| \int_a^b f(t)dt \right| \leq \int_a^b |f(t)|dt $$
\end{lemma}

Proof: Since $|f|$ is continuous, it is Riemann integrable. Let $r \in [0, \infty)$ and $\theta \in \mathbb{R}$ such that $\int_a^b f(t)dt = r(\cos \theta + i \sin\theta)$, and define $\zeta = \cos \theta - i \sin \theta$. Then 
$$ \left| \int_a^b f(t)dt \right| = \zeta \int_a^b f(t)dt = \re (\zeta \int_a^b f(t)dt) = \int_b^a \re (\zeta f(t))dt $$
$$ \leq \int_a^b \left| \re (\zeta f(t))\right|dt \leq \int_a^b |\zeta f(t)|dt = \int_a^b |f(t)|dt $$

\section{Analytic Functions}
\subsection{Power Series} 
Let $a \in \mathbb{C}$. Then a power series about $a$ is a series of the form 
$$ \sum \alpha_n(z - \alpha)^n $$
where $\alpha_k \in \mathbb{C}$, $\forall k \in \mathbb{N}_0$. For each $z \in \mathbb{C}$, the series might or might not converge.

\vspace{5mm}
\noindent
If $U \in \mathbb{C}$ is open, $f: U \to \mathbb{C}$ is infinitely differentiable (smooth) then 
$$ \sum \frac{f^{(n)}(a)}{n!} (z - a)^n $$
is called the power series of $f$ about $a \in U$.

\vspace{5mm}
\noindent
Define $R \in [0, \infty]$ the supremum of all $r \in [0, \infty)$ such that the series 
$$ \sum |\alpha_n|r^n $$
is convergent. Call $R$ the radius of convergence of the power series.

\begin{lemma}
Let $R$ be the radius of convergence of $\sum \alpha_k (z - a)^k$. Then the following are equivalent
\begin{enumerate}[(I)]
\item If $z \in \mathbb{C}$ with $|z - a| < R$ then the series converges
\item If $z \in \mathbb{C}$ with $z \notin \overline{B_R(a)}$, then the series diverges 
\item If $0 < r < R$ then $\sum \alpha_n (z-a)^n$ and $\sum |\alpha_n(z-a)^n|$ are convergent uniformly 
\end{enumerate}
\end{lemma}

Proof: WLOG assume $a = 0$. We will prove (II) and (III), and (I) will follow from (III).

We will prove (II) via contraposition. Let $z \in \mathbb{C} - \{ 0 \}$ and suppose that $\sum \alpha_n z^n$ is convergent. Thus $\{ \alpha_n z^n : n \in \mathbb{N} \}$ is bounded. Let $M > 0$ be such that $|\alpha_n z^n| \leq M$ for all $n \in \mathbb{N}_0$. Let $r \in (0, |z|)$, then 
$$ |\alpha_n| r^n \leq M\left(\frac{r}{|z|}\right)^n, \forall n \in \mathbb{N}_0 $$
Since $\frac{r}{|z|} < 1$, the series $\sum |\alpha_n|r^n$ converges, by definition of radius of convergence, $r < R$, so $|z| \leq R$.

To prove (III), let $r \in (0, R)$, so $\sum |\alpha_n|r^n$ is convergent. If $z \in \overline{B_r(a)}$ then $|\alpha_n z^n| \leq |\alpha_n| r^n$ for all $n \in \mathbb{N}$. Thus by Weierstrass, we get that $\sum \alpha_n z^n$ and $\sum |\alpha_n z^n|$ are uniformly convergent for all $z \in \overline{B_r(0)}$.

\begin{example}
The radius of convergence of $\sum z^n$ is 1, so it is uniformly convergent on $\overline{B_r(0)}$, $\forall r \in (0, 1)$
\end{example}

\begin{definition}
Let $U \subset \mathbb{C}$ be open and $f: U \to \mathbb{C}$ and $a \subset U$. Then $f$ is analytic at $a$ if $\exists R > 0$ and a power series $\sum \alpha_n(z-a)^n$ about $a$ such that $B_R(a) \subset U$, the power series has positive radius of convergence at least $R$, and 
$$ f(z) = \sum_{n = 0}^{\infty} \alpha_n(z - a)^n, \hspace{3mm} \forall z \in B_R(a) $$
We say that $f$ is analytic if it is analytic $\forall a \in D$.
\end{definition}

Note that every polynomial is analytic.

\begin{lemma}
The power series $\sum \alpha_n (z -a)^n$ and $\sum n\alpha_n(z - a)^{n-1}$ have the same radius of convergence.
\end{lemma}

Proof: WLOG assume $a = 0$. Let $R, \hat{R}$ be the respective radii of convergence. Let $r \in [0, R)$. Then there exists $\rho \in \mathbb{R}$ with $r < \rho < R$ such that $\sum |\alpha_n|\rho^n$ is convergent and $\lim_{n \to \infty} n \left(\frac{r}{\rho}\right)^n = 0$. Then $\exists M \in \mathbb{R}$ such that $n \left(\frac{r}{\rho}\right)^n \leq M$ for all $n \in \mathbb{N}_0$. Then 
$$ n|\alpha_n|r^n = n \left(\frac{r}{\rho}\right)^n|\alpha_n| \rho^n \leq M |\alpha_n| \rho^n, \hspace{3mm} \forall n \in \mathbb{N}$$
Thus since $\sum |\alpha_n| \rho^n$ converges, so too does $\sum n|\alpha_n|r^n$ (and thus $\sum n|\alpha_n|r^{n-1}$). Thus $\hat{R} \geq r$ and so $\hat{R} \geq R$. The other direction is similar.

\begin{theorem}
Let $R \in (0, \infty)$ and suppose the radius of convergence of the power series $\sum \alpha_n(z-a)^n$ is also at least $R$. Define $f:B_R(a) \to \mathbb{C}$ as $f(z) = \sum_{n=0}^{\infty} \alpha_n(z - a)^n$. Then $f$ is holomorphic (differentiable). Moreover, $f'(z) = \sum_{n = 0}^{\infty} n\alpha_n (z - a)^n$, $\forall z \in B_R(a)$. Hence $f$ is infinitely differentiable and $\alpha_n = \frac{f^{(n)}(a)}{n!}$ for all $n \in \mathbb{N}_0$.
\end{theorem}

Proof: WLOG let $a = 0$. Fix $z_0 \in B_{R}(0)$, and let $\epsilon > 0$. Fix an $r \in (|z_0|, R)$. By lemma 2.4, $\exists n \in \mathbb{N}$ such that 
$$ \sum_{n = N+1}^{\infty} n|\alpha_n|r^{n-1} \leq \frac{\epsilon}{4} $$
Then $\forall z \in B_r(0) - \{ z_0 \}$, we have 
$$ \left| \frac{f(z) - f(z_0)}{z - z_0} - \sum_{n=1}^{\infty}n\alpha_nz_0^{n-1} \right| = \left| \sum_{n =0}^{\infty} \alpha_n\left(\frac{z^n - z_0^n}{z - z_0} - nz_0^{n-1}\right)\right| $$
$$ = \left| \sum_{n=1}^{\infty} \alpha_n\left((\sum_{k=0}^{n-1} z^k z_0^{n - 1 - k}) - nz_0^{n-1}\right)\right|$$
$$ \leq \left| \sum_{n=1}^{N}\alpha_n \sum_{k = 0}^{n-1}(z^k z_0^{n-1-k} - z_0^{n-1}) \right| + \sum_{n = N+1}^{\infty} 2n|\alpha_n|r^{n-1} $$
The right sum is bounded by $\frac{\epsilon}{2}$. Note that $z \mapsto z^kz_0^{n-1-k} - z_0^{n-1}$ is continuous at $z_0$, with value $0$ for all $n \in \{1, ..., N\}$ and for all $k \in \{0, ..., n-1 \}$, so we can bound it and get from the above expression
$$ \lim_{z \to z_0} \left( \frac{f(z) - f(z_0)}{z - z_0} - \sum_{n=1}^{\infty} \alpha_n z_0^{n-1}\right) = 0$$
Thus $f$ is differentiable at $z_0$ with $f'(z_0) = \sum_{n=1}^{\infty} \alpha_n nz_0^{n-1}$. The proof that $\alpha_n = \frac{f(n)(a)}{n!}$ follows from repeated differentiation.

\begin{corollary}
Every analytic function is differentiable.
\end{corollary}

\begin{corollary} Let $\sum \alpha_n(z-a)^n$ and $\sum \beta_n (z-a)^n$ be two power series with radii of convergence $R_{1}, R_2$ respectively. Let $\epsilon \leq \min(R_1, R_2)$. Suppose that $\sum_{n=1}^{\infty}\alpha_n(z - a)^n = \sum_{n=1}^{\infty} \beta_n (z-a)^n$ for all $z\in B_{\epsilon}(a)$. Then $\alpha_n = \beta_n$ for all $n \in \mathbb{N}$.
\end{corollary}

Proof: Consider $\sum(\alpha_n - \beta_n)(z-a)^n$. It is identically $0$, so the $n^{th}$ order derivatives are $0$, so $\alpha_n = \beta_n$.

\vspace{5mm}
\noindent
Define 
$$ e^{x} = \sum_{n=0}^{\infty} \frac{x^n}{n!}$$
$$ \sin(x) = \sum_{n = 0}^{\infty} (-1)^n \frac{x^{2n+1}}{(2n+1)!}$$
$$ \cos(x) = \sum_{n=0}^{\infty} (-1)^n \frac{x^{2n}}{(2n)!} $$
These have infinite radius of convergence, and so we take these as a definition for functions $\mathbb{C} \to \mathbb{C}$. Note that we can show 
$$ e^{iz} = \sum i^n \frac{z^n}{n!} = ... = \cos{z} + i\sin{z} $$
We also have the formulas
$$ \cos(z) = \frac{e^{iz} + e^{-iz}}{2}, \hspace{3mm} \sin(z) = \frac{e^{iz} - e^{-iz}}{2} $$

\vspace{5mm}
\noindent
Let $w, z\in\mathbb{C}$ and $f:\mathbb{R} \to \mathbb{C}$ such that $f(t) = e^{-tz}e^{w + tz}$. We can verify by the product rule that $f' = 0$, and in particular $f(1) = f(0)$. Thus $e^{-z}e^{w+z} = e^{w}$. Let $a, b\in\mathbb{C}$, and choose $z = -a$, $w = a+b$. Then $e^a e^b = e^{a+b}$.

\vspace{5mm}
\noindent
Let $x, y, u, v \in \mathbb{R}$. Suppose $e^{x + iy} = e^{u + iv}$. Thus $e^{x} = e^u$, so $x = u$. This implies $e^{iy} = e^{iv}$, so $y-v \in 2\pi\mathbb{Z}$. Thus if $z,w \in \mathbb{C}$, then $e^z = e^w$ if and only if $z - w \in 2\pi i\mathbb{Z}$.

\subsection{Curves}
\begin{definition} 
A curve is a continuous map $\gamma:[a,b] \to \mathbb{C}$ where $a < b \in \mathbb{R}$. Let $\gamma(a)$ be the initial point and $\gamma(b)$ be the final point. Let $\gamma*$ denote the image of $\gamma$ in $\mathbb{C}$. 
\end{definition}

If $A \subset \mathbb{C}$, we say $\gamma$ is in $A$ if $\gamma* \subset A$. Lets say $\gamma$ is closed if $\gamma(a) = \gamma(b)$. 

$\gamma$ is called smooth if it is $\mathbb{C}^1$ continuously differentiable. There is also a notion of being piecewise smooth (it is smooth every except at a finite number of points).

If $\gamma:[a,b]\to\mathbb{C}$ and $\mu:[c,d]\to\mathbb{C}$ with $\gamma(b) = \mu(c)$ then we define the combined curve $\gamma \oplus \mu:[a, b+ d - c] \to \mathbb{C}$ as 
$$ (\gamma \oplus \mu)(t) = \begin{cases} \gamma(t) & t \in [a,b] \\ \mu(c + t - b) & t \in (b, b + d - c] \end{cases}$$

\begin{example}
Define $\gamma : [-1, 1]\to\mathbb{C}$ by 
$$ \gamma(t) = \begin{cases} t^2(1+i) & t \in [0, 1] \\ t^2(-1 + i) & t \in [-1, 0) \end{cases} $$
is smooth. But 
$$ \gamma(t) = \begin{cases} t(1+i) & t \in [0, 1] \\ 2t(1 + i) & t\in[-1, 0)\end{cases} $$
is not smooth.
\end{example}

\vspace{5mm}
\noindent

Since $\gamma$ is continuous and $[a,b]$ is sequentially compact, it follows that $\gamma*$ is sequentially compact (and in particular it is closed).

\begin{definition}
Let $\gamma :[a,b] \to \mathbb{C}$ be a piecewise smooth curve, $D \subset \mathbb{C}$, and if $f: D\to\mathbb{C}$ a continuous function. Suppose $\gamma* \subset D$. If $\gamma$ is smooth, we define the contour integral of $f$ along $\gamma$ to be 
$$ \int_{\gamma} f(z)dz = \int_a^b f(\gamma(t))\gamma'(t)dt $$
If $\gamma$ is only piecewise smooth, with $a = s_0 < s_1 < ... < s_n = b$ and $\gamma|_{[s_{k-1}, s_k]}$ is smooth, then 
$$ \int_{\gamma} f(z)dz = \sum_{i=1}^{n} \int_{\gamma_{[s_{i-1}, s_i]}}f(z)dz $$
\end{definition}

The lengths of curve $\gamma$ is given by 
$$ l(y) = \int_a^b |\gamma'(t)|dt $$
For piecewise $\gamma$,
$$ l(\gamma) = \sum_{i=1}^{n} l(\gamma_i) $$

\begin{example}
Define $\gamma:[0. 2\pi] \to \mathbb{C}$, as $t \mapsto e^{it}$. Moreover define $f: \mathbb{C}- \{0\} \to \mathbb{C}$ to be $f(z) = \frac{1}{z}$. 
$$ \int_{\gamma} f(z)dz = \int_0^{2\pi} \frac{1}{e^{it}} * ie^{it}dt = i \int_0^{2\pi}dt = 2\pi i $$
\end{example}

\begin{lemma} 
Let $\gamma:[a,b] \to \mathbb{C}$ be piecewise smooth, and $f, \tilde{f}:\gamma* \to \mathbb{C}$. Then we have 
\begin{enumerate}[(I)] 
\item $\int_{\gamma} (f + \tilde{f})(z)dz = \int_{\gamma} f(z)dz + \int_{\gamma} \tilde{f}(z)dz $
\item $\int_{\gamma} \lambda f(z)dz = \lambda \int_{\gamma}f(z)dz $
\item $|\int_{\gamma} f(z)dz | \leq Ml(\gamma)$ where $M = \sup\{|f(z)|: z \in \gamma*\}$
\item Let $c, d \in \mathbb{R}$, $c<d$ and $\varphi:[c,d] \to [a,b]$ be a continuously differentiable function such that $\varphi(c) = a$, and $\varphi(d) = b$. Suppose $\gamma$ is smooth or $\varphi$ is strictly increasing. Then $\gamma \circ \varphi$ is piecewise smooth and $\int_{\gamma}f(z)dz = \int_{\gamma \circ \varphi} f(z)dz$
\item $f_k:\gamma* \to \mathbb{C}$ be continuous functions with $\lim f_k = f$ uniformly on $\gamma*$. Then 
$$ \lim_{n \to \infty} \int_{\gamma}f_n(z)dz = \int_{\gamma} f(z)dz $$
\item If $\gamma_1, \gamma_2$ are curves with $\gamma = \gamma_1 \oplus \gamma_2$, then 
$$ \int_{\gamma} f(z)dz = \int_{\gamma_1} f(z)dz + \int_{\gamma_2}f(z)dz $$
\end{enumerate} 
\end{lemma}

Proof of (IV): Suppose $\gamma$ is smooth, then 
$$ \int_{\gamma \circ \varphi} f(z)dz = \int_{c}^d f(\gamma \circ \varphi(t)) (\gamma \circ \varphi)'(t)dt = \int_c^d f(\gamma(\varphi(t))) \gamma'(\varphi(t)) \varphi'(t)dt $$
$$ = \int_a^b f(\gamma(\tilde{t})) \gamma'(\tilde{t})d\tilde{t} = \int_{\gamma} f(z)dz $$

\begin{proposition}
Let $U \subset \mathbb{C}$ be open, $f: U \to \C$ continuously differentiable, $\gamma : [a, b] \to \C$ be piecewise smooth with $\gamma^* \subset U$. Then 
$$ \int_{\gamma} f'(z)dz = f(\gamma(b)) - f(\gamma(a)) $$
\end{proposition}

Proof: Suppose $\gamma$ is smooth (proof is almost the same). Then we have that 
$$ \int_{\gamma} f'(z)dz = \int_a^b f'(\gamma(t))\gamma'(t)dt = \int_a^b (f \circ \gamma)'(t) dt = f(\gamma(b)) - f(\gamma(a)) $$
where the last equality follows from FTC.

\begin{corollary}
Let $U \subset \C$ be open with $f: U \to \C$ be holomorphic and $f' = 0$. Suppose $\forall z_1, z_2 \in U$, there exists $\gamma :[0, 1] \to U$ with $\gamma(0) = z_1$ and $\gamma(1) = z_2$ where $\gamma$ is smooth. Then $f$ is constant.
\end{corollary}

\begin{remark}
The above requirement is for $f$ to be constant when $f' = 0$ also includes that $U$ must be path connected. In general $f' = 0$ does not imply $f$ is constant.
\end{remark}

\begin{theorem}
Suppose $\gamma:[a,b] \to \C$ be piecewise smooth and $g: \gamma^* \to \C$ a continuous function. Let $U = \C - \gamma*$ (it is open) and define $f: U \to \C$ as 
$$f(z) = \int_{\gamma} \frac{g(w)}{w- z} dw$$
Then $f$ is analytic. More specifically, let $z_0 \in U$, and 
$$R = \inf\{ |w - z_0| : w \in \gamma*\}$$
Then $R > 0$ and $\forall n \in N$ let $\alpha_n = \int_{\gamma} \frac{g(w)}{(w - z_0)^{n+1}} dw$. Then the power series $\sum \alpha_n(z - z_0)^n$ has a radius of convergence $> R$, and 
$$ f(z) = \sum_{n=0}^{\infty} \alpha_n (z - z_0)^n $$
\end{theorem}

Proof: Note $\gamma*$ is closed since $\gamma$ is continuous, so $U$ is open. Let $z_0$ be given and $R$ defined as above. Let $z \in B_R(z_0)$. For $w \in \gamma*$, 
$$ \left| \frac{z - z_0}{w - z_0} \right| \leq \frac{|z - z_0|}{R} < 1 $$
Therefore, 
$$ \frac{1}{w - z} = \frac{1}{(w - z_0) - (z - z_0)} = \frac{1}{w - z_0} * \frac{1}{1 - \frac{z - z_0}{w - z_0}} $$
$$ = \frac{1}{w- z_0} \sum_{n=0}^{\infty} \left(\frac{z - z_0}{w - z_0}\right)^n $$
since the series is absolutely convergent for $z \in B_R(z_0)$, and thus is convergent with the given formula. It is also true $\forall w in \gamma^*$. Now define $h, h_0, h_1, ...: \gamma* \to \C$ as 
$$ h(w) = \frac{g(w)}{w-z}, \hspace{3mm} h_n(w) = \frac{g(w)(z - z_0)^n}{(w - z_0)^{n+1}} $$
then on $\gamma^*$, $\lim_{n \to \infty} \sum_i^n h_i = h$. By Lemma 2.12 (VI) we have that 
$$ f(z) = \int_{\gamma} h(w)dw = \sum_{n=0}^{\infty} \int_{\gamma}h_n(w)dw = \sum_{n = 0}^{\infty}(z - z_0)^n \int_{\gamma} \frac{g(w)}{(w - z_0)^{n-1}} dw $$
$$ = \sum_{n = 0}^{\infty} \alpha_n (z - z_0)^n $$
In particular, the series $\sum \alpha_n (z - z_0)^n$ converges $\forall z \in B_R(z_0)$. By Lemma 2.1, the power series has radius of convergence at least $R$.

\begin{definition}
Let $\gamma$ be a closed piecewise smooth curve. Define on $\C - \gamma^*$
$$ \Ind_{\gamma}: \C - \gamma^* \to \C $$
$$ z \mapsto \frac{1}{2\pi i}\int_{\gamma} \frac{1}{w-z} dw $$
This is called the index function of $\gamma$ with respect to $z$.
\end{definition}

Note that $\Ind_{\gamma}$ is analytic. 

\begin{proposition}
Let $\gamma$ be a piecewise smooth closed curve. Then $\Ind_{\gamma}(z) \in \mathbb{Z}$ for all $z \in \C - \gamma^*$. Moreover there exists $R > 0$ such that $\Ind_{\gamma}(z) = 0$ for all $z \in \C - B_R(0)$
\end{proposition}

Proof: Let $\gamma:[a,b] \to \C$ be smooth. Define $f:[a, b] \to \C$ as $t \mapsto \int_a^t \frac{\gamma'(s)}{\gamma(s) - z} ds$. Then $f$ is differentiable and $f'(t) = \frac{\gamma'(t)}{\gamma(t) - z}$. Define $g:[a, b] \to \C$ by $g(t) = e^{-f(t)}(\gamma(t) - z)$. Then $g'(t) = 0$ (easily verifiable) and $[a,b]$ is path connected, so $g$ is constant. Therefore 
$$ \frac{e^{f(t)}}{e^{f(a)}} = \frac{\gamma(t) - z}{\gamma(a) - z} $$
Since $f(a) = 0$ and $\gamma(b) = \gamma(a)$, it follows that $e^{f(b)} = 1$ so $f(b) \in 2\pi i \mathbb{Z}$, but $f(b) = 2\pi i \Ind_{\gamma}(z)$, so we have $\Ind_{\gamma}(z) \in \mathbb{Z}$.

Since $\gamma^*$ is bounded, $\exists r > 0$ such that $\gamma^* \subset B_r(0)$ and we have that 
$$ |\Ind_{\gamma}(z)| \leq \frac{Kl(\gamma)}{R - r} $$
where $R > r$ and $z \in \C - B_R(0)$ and $K$ is a constant in $\mathbb{R}$. Since $\Ind_{\gamma}(z) \in \mathbb{Z}$, for large enough $R$ we have $\Ind_{\gamma}(z) = 0$. Since $\Ind_{\gamma}(z) \in \mathbb{Z}$, we conclude for sufficiently large $R$, we conclude $\Ind_{\gamma}(z) = 0 $.

\begin{corollary}
Let $\gamma$ be a closed piecewise smooth curve, and $\varphi:[0, 1] \to \C - \gamma^*$ be continuous. Then $\Ind_{\gamma}(\varphi(0)) = \Ind_{\gamma}(\varphi(1))$.
\end{corollary}

\begin{definition}
For all $z_0 \in \C$, $r>0$ define $\gamma:[0, 2\pi] \to \mathbb{C}$, as 
$$ t \mapsto z_0 + re^{it} $$
Then denote this curve by $\Gamma_r(z_0) = \gamma$.
\end{definition}

\begin{example}
Let $z_0 \in \C$, $r > 0$. Then we have $\Ind_{\Gamma_r(z_0)}(z_0) = 1$. moreover, if $|z - z_0| > 3r$, then estimates show 
$$ |\Ind_{\Gamma_r(z_0)}(z)| \leq \frac{1}{2} $$
and by Corollary 2.18 we have that 
$$ \Ind_{\Gamma_r(z_0)}(z) = \begin{cases} 1 & |z - z_0| < r \\ 0 & |z - z_0| > r \end{cases} $$
\end{example}

\subsection{Cauchy's Theorem for a triangle and a convex set}
\begin{definition} 
Let $z_1, z_2 \in \C$. Denote by $[z_1, z_2]$ the curve $\gamma:[0, 1] \to \C$ which maps $t \mapsto z_1 + t(z_2 - z_1)$. Let $z_3 \in C$. Then define 
$$ \Delta(z_1, z_2, z_3) = \{ t_1z_1 + t_2z_2 + t_3z_3 | t_i \geq 0, \sum_i t_i = 1\} $$
Denote also by $\partial \Delta(z_1, z_2, z_3)$ the piecewise smooth curve $\gamma:[0,3] \to \C$ by 
$$ \gamma = [z_1, z_2] \oplus [z_2, z_3] \oplus [z_3, z_1] $$
\end{definition}

\begin{remark}
$$ \int_{\partial\Delta} = \int_{[z_1, z_2]} + \int_{[z_2, z_3]} + \int_{[z_3, z_1]} $$
If $z_1, z_2, z_3$ are collinear, then $\int_{\partial\Delta} = 0$.
\end{remark}

\begin{theorem}
Cauchy-Goursat Theorem: Let $U \subset \C$ open, $p \in U$, $\Delta \subset U$, $f: U \to \C$ be continuous, and suppose $f: U - \{p\}$ be holomorphic. Then $\int_{\partial\Delta}f(z)dz = 0$.
\end{theorem}

Proof: There are 3 steps depending on where $p$ is relative to $\partial\Delta$.

Step 1: Suppose $p \notin \Delta$. Then denote the midpoint of the sides of the triangle by $z_1', z_2', z_3'$ where $z_1' = \frac{z_2 + z_3}{2}$, and the others are defined similarly. Connect the $z_1', z_2', z_3'$ in the original triangle, and so we get 4 triangles. Label them $\Delta^(i)$ arbitrarily. Note that we have 
$$ \int_{\partial\Delta} f(z)dz = \sum_{k=1}^4 \int_{\partial\Delta^(k)} f(z)dz $$
so 
$$ \int_{\partial\Delta} \leq \left|\int_{\partial\Delta} f(z)dz\right| \leq \sum_{k = 1}^4 \left|\int_{\partial\Delta^{(k)}} f(z)dz\right| \leq 4\left| \int_{\partial\Delta^{(i)}} f(z)dz \right|$$
for some $i = 1, 2, 3, 4$. Let $\Delta_1 = \Delta^{(i)}$, and note that $l(\partial\Delta_1) = \frac{1}{2}l(\partial\Delta)$. By induction we can continue and make a sequence of closed triangles $\Delta_2, \Delta_3, ...$ with $\Delta_{k+1} \subset \Delta_{k}$ and 
$$ \left|\int_{\partial\Delta_k} f(z)dz \right| \leq 4\left|\int_{\partial\Delta_{k+1}} f(z)dz\right| $$
$$l(\Delta_{k+1}) = \frac{1}{2} l(\Delta_{k})$$
Note that for all $z \in \partial\Delta_n$, $|z - z_0| < l(\partial\Delta_n)$.

Thus
$$ \left|\int_{\partial\Delta} f(z)dz \right| \leq 4^n\left|\int_{\partial\Delta_{n}} f(z)dz\right| $$
$$l(\Delta_{n}) = \frac{1}{2^n} l(\Delta)$$
By compactness, there exists $z_0 \in \cap_{n=1}^{\infty}\Delta_n$ and $z_0 \neq p$. So $f$ is differentiable at $z_0$ and so 
$$ f(z) = f(z_0) + f'(z_0)(z - z_0) + \psi(z)(z - z_0) $$
with $\psi:U\to\C$ continuous at $z_0$, and $\psi(z_0) = 0$. The map 
$$ z \mapsto f(z_0) + f'(z_0)( z - z_0) $$
comes from the derivative of the map 
$$ z \mapsto f(z_0)z + \frac{1}{2}f'(z_0)(z - z_0)^2 $$
Thus for all $n$,
$$ \int_{\partial\Delta_n}f(z_0) + f'(z_0)(z - z_0)dz = 0 $$
Therefore we get that 
$$ \left| \int_{\partial\Delta_n} f(z)dz \right| = \left| \int_{\partial\Delta_n} f(z) - (f(z_0) + f'(z_0)(z - z_0))dz \right| = \left| \int_{\partial\Delta_n}(z - z_0)\psi(z)dz \right| $$
$$ \leq l(\partial\Delta_n)\sup\{|(z - z_0)\psi(z)| : z \in\partial\Delta_n\} $$
$$ \leq (l(\partial\Delta_n))^2\sup\{|\psi(z)|:z \in \partial\Delta_n\} $$
Since $\psi$ is continuous at $z_0$ with $\psi(z_0) = 0$, for all $\epsilon > 0$ we have $\exists \delta > 0$ such that $|\psi(z)| < \epsilon$ for $z \in B_{\delta}(z_0) \cap U$. Since $\lim_{n \to \infty}l(\partial\Delta_n) = 0$ (then the diameter also goes to 0) we have that $\exists n \in \mathbb{N}$ such that $\bar{\Delta_n} \subset B_{\delta}(z_0)$. Then $\sup\{ |\psi(z)|: z \in \partial\Delta_n\} \leq \epsilon$ and 
$$ \left| \int_{\partial\Delta_n} f(z)dz \right| \leq \epsilon(l(\partial\Delta))^2 = \frac{\epsilon}{4^n}$$
Hence 
$$ \left|\int_{\partial\Delta}f(z)dz \right| \leq 4^n \frac{\epsilon}{4^n} = \epsilon $$
and we conclude that 
$$ \int_{\partial\Delta}f(z)dz = 0 $$

In the next step, suppose $p$ is a vertex. Then WLOG let $\Delta = \Delta(p, z_2, z_3)$. Let $\epsilon \in (0, 1)$ and set $p_2 = \epsilon z_2 + (1-\epsilon)p$ and $p_3 = \epsilon z_3 + (1- \epsilon)p$. Then we have that 
$$ \int_{\partial\Delta} = \int_{\partial\Delta(p, p_2, p_3)} + \int_{\partial\Delta(p_2, z_2, z_3)} + \int_{\partial\Delta(p_3, p_2, z_3)} $$
We have that the $2^{nd}$ and $3^{rd}$ integrals are zero by the earlier case, so we only need to consider the first integral. Denoting $\partial\Delta(p_1, p_2, p_3)$ by $\partial\Delta p$ we have that 
$$ \left|\int_{\partial\Delta p}f(z)dz\right| \leq l(\partial\Delta_p) \sup\{|f(z)|:z \in \partial\Delta_p\}$$
There exists $r > 0$ such that $\bar{B_r(p)} \subset U$ (by openness of $U$), and since $f$ is continuous $\exists M > 0$ such that $|f(z)| < M$ for all $z \in \bar{B_r(p)}$. If $\epsilon$ is small enough, then $\partial\Delta_p \subset B_r(p)$ and since $\lim_{\epsilon \to 0}l(\partial\Delta_p) = 0$, we get that 
$$ l(\partial\Delta_p)\sup\{|f(z)|:z \in \partial\Delta_p\} \leq \epsilon' $$
proving the desired.

For the case that $p$ is not a vertex but $p \in \Delta$, consider the integral 
$$ \int_{\partial\Delta(z_1, z_2, z_3)} = \int_{\partial\Delta(z_1, z_2, p)} + \int_{\partial\Delta(p, z_2, z_3)} + \int_{\partial\Delta(z_3, z_1, p)} $$
All three of the integrals are $0$ by case 2, and we are done.

We define a set $A \subset \mathbb{C}$ to be convex if 
$$ tz_1 + (1-t)z_2 \in A $$ 
for all $z_1, z_2 \in A$, $t \in [0, 1]$. 

\begin{proposition}
Let $U \subset \mathbb{C}$ be open and convex, $f: U \to \C$ continuous such that 
$$ \int_{\partial\Delta} f(z)dz = 0 $$
for all $\Delta \subset U$. Then $\exists F: U \to C$ holomorphic such that $F' = f$.
\end{proposition}

Proof: Fix $a \in U$ and define $F: U \to \C$ by $F(z) = \int_{[a,z]}f(w)dw$. Fix $z_0 \in U$ and let $z \in U$. Then we have that 
$$ 0 = \int_{\partial\Delta(a, z, z_0)} f(w)dw = \int_{[a,z]}f(w)dw + \int_{[z, z_0]}f(w)dw + \int_{[z_0, a]}f(w)dw $$
$$ = F(z) + \int_{[z, z_0]}f(w)dw - F(z_0) $$
Now let $\epsilon > 0$, since $f$ is continuous at $z_0$ we have $\exists \delta >0$ such that $|f(z) - f(z_0)| < \epsilon$, for all $|z - z_0| < \delta$. We can write 
$$ \frac{F(z) - F(z_0)}{z - z_0} - f(z_0) = \frac{1}{z - z_0} \int_{[z_0, z]} (f(w) - f(z_0))dz $$ 
and so for all $z \in B_r(z_0)$,
$$ \left| \frac{F(z) - F(z_0)}{z - z_0} - f(z_0)\right| \leq \frac{1}{|z - z_0|} \int_{[z_0, z]} |f(w) - f(z_0)|dw \leq \frac{1}{|z - z_0|} l([z_0, z]) \epsilon \leq \epsilon $$
for all $z in U$ $\delta$ close to $z_0$. Thus $F'(z_0) = f(z_0)$.

\begin{corollary}
Let $U$ be open and convex with $p \in U$, $f:U\to\C$ continuous who is holomorphic off $p$. Then $f = F'$ for some holomorphic function $F: U \to \mathbb{C}$. Explicitly, $\forall a \in U$, we have that $F(z) = \int_{[a,z]} f(w)dw$.
\end{corollary}

Proof: Because $f$ is holomorphic off $p$, by Cauchy-Goursat, $\int_{\Delta} f(z)dz = 0$ for all $\Delta \in U$, and thus by Theorem 2.23 we have $F$ as defined in the proof of the theorem.

\begin{theorem}
Cauchy's theorem for a convex set. Let $U$ be convex and open, and $p \in U$. Suppose $f:U \to \C$ continuous on $U$ and holomorphic on $U - \{p\}$. Then $\int_{\gamma}f(z)dz = 0$ where $\gamma$ is a closed piecewise smooth curve inside of $U$.
\end{theorem}

Proof: $\exists F$ such that 
$$ \int_{\gamma}f(z)dz = \int_{\gamma}F'(z)dz = 0 $$

\subsection{Holomorphicity implies Analycity}

\begin{theorem}
Let $U$ be open and convex., $\gamma$ be piecewise smooth closed, and $\gamma* \subset U$. $f: U \to \C$ holomorphic. Then $\forall z \in U - \gamma^*$, we have 
$$ f(z)\Ind_{\gamma}(z) = \frac{1}{2\pi i}\int_{\gamma}\frac{f(w)}{w - z}dw $$
\end{theorem}

Proof: Let $z \in U - \gamma^*$, and define 
$$ g(w) = \begin{cases} \frac{f(w) - f(z)}{w - z} & w \in U - \{z\} \\ f'(z) & w = z \end{cases} $$
Note that $g$ is continuous and holomorphic on $U - \{z\}$. By Theorem 2.25,
$$ 0 = \frac{1}{2\pi i} \int_{\gamma}g(w)dw = \frac{1}{2\pi i} \int_{\gamma} \frac{f(w)}{w-z}dw - \frac{1}{2\pi i}\int_{\gamma}\frac{f(z)}{w-z}dw $$
as desired.

\begin{theorem}
Every holomorphic function is analytic. Stronger, $U$ is open, $f: U \to \C$ is holomorphic. Let $z_0 \in U$, $r > 0$ such that $B_r(z_0) \subset U$. Then there exists power series $\sum \alpha_n(z - z_0)^n$ which has radius of convergence at least $r$, and $f(z) = \sum_{n=0}^{\infty}\alpha_n(z - z_0)^n$ for all $z \in B_r(z_0)$.
\end{theorem}

Proof: Consider $\rho \in (0, r)$. Then $\Gamma_{\rho}(z_0) \subset U$ and $\forall z \in B_{\rho}(z_0)$ we have $\Ind_{\Gamma_{\rho}(z_0)}(z) = 1$. Then restrict $f$ to $B_{\frac{r + \rho}{2}}(z_0)$ and we get that 
$$ f(z) = \frac{1}{2\pi i} \int_{\Gamma_{\rho}(z_0)} \frac{f(w)}{w - z} dw $$
for all $z \in B_{\rho}(z_0)$. Consider the function 
$$ z \mapsto_{\Gamma_{\rho}(z_0)} \frac{f(w)}{w-z} dw $$
Theorem 2.15  says this function is analytic. Thus it is infinitely differentiable on $B_{\rho}(z_0)$ and the power series $\sum \frac{f^{(n)}(z_0)}{n!} (z - z_0)^n$ has radius of convergence of at least $\rho$. Thus we get that it has radius of convergence $\geq r$ and 
$$ f(z) = \sum_{n=0}^{\infty} \frac{f^{(n)}(z_0)}{n!} (z - z_0)^n $$

\begin{corollary} 
$f$ holomorphic implies that $f'$ is holomorphic.
\end{corollary}

\begin{theorem}
Morera's Theorem: Let $U$ be open, $f:U\to\C$ continuous. Then $f$ is holomorphic if and only if $\forall \Delta \subset U$ we have 
$$ \int_{\partial\Delta}f(z)dz = 0 $$
\end{theorem}

Proof: The forward is obvious from Cauchy-Goursat. For the reverse, it is sufficient to prove it for an open ball. Suppose $U$ is convex, then by Theorem 2.23, $f = F'$ on the convex set. Then applying the Fundamental Theorem of Calculus and using the fact that $\partial\Delta$ is closed, we get the desired result.

\begin{lemma}
Let $a\in U$ which is open, and $f: U \to \mathbb{C}$ continuous on $U$ and holomorphic off $a$. Then $f$ is holomorphic.  
\end{lemma}

\subsection{Estimates and Consequences}
\begin{lemma}
Let $U \subset \C$ be open, $a \in U$, and $r > 0$. Then $\overline{B_r(a)} \subset U$ if and only if $\exists R \in (r, \infty)$ such that $B_R(a) \subset U$.
\end{lemma}

\begin{proposition}
Let $U$ be open, $f: U \to \C$ holomorphic, $a \in U$, $r > 0$, such that $\overline{B_r(a)} \subset U$. Then 
$$ f^{(n)}(z) = \frac{n!}{2\pi i} \int_{\Gamma_r(a)} \frac{f(z)}{(w-z)^{n+1}} dw $$
\end{proposition}

Proof: Cauchy's Formula (Theorem 2.26) gives 
$$ f(z) = \frac{1}{2\pi i} \int_{\Gamma_r(a)} \frac{f(w)}{w-z} dw $$
for all $z \in B_r(a)$. Then by Theorem 2.15 the result is true. 

\begin{corollary}
Cauchy's Inequality: Let $U$, $f: U \to \C$, $a$, $r$ defined as above. Then we have that 
$$ |f^{(n)}(z)| \leq n! \frac{r}{(r - |z - a|)^{n+1}} \max_{w \in \partial B_r(a)} |f(w)| $$
Moreover if $z \in B_{\frac{r}{2}}(a)$, then 
$$ |f^{(n)}(z)| \leq \frac{n!2^{n+1}}{r^n} \max_{w \in \partial B_r(a)} |f(w)| $$
\end{corollary}

Proof: By Theorem 2.22, the fact that for all $w\in\partial B_r(a)$, 
$$|w - z| \geq r - |z - a|$$
 and lemma 2.12(III) this is true.
 
 \begin{definition}
 A holomorphic function whose domain in $\C$ is called an "entire" function.
 \end{definition}
 
 \begin{theorem}
 Liouville: Every bounded holomorphic function $f$ is constant.
 \end{theorem}
 
 Proof: Let $f$ be entire, bounded by $M$. It follows from Corollary 2.23 that 
 $$ f'(a) \leq \frac{M}{r} $$
 for all $a \in C$, $r > 0$. Therefore $f' = 0$ and vanishes on the connected set $\C$ and thus is constant.
 
 \begin{lemma}
 Let $f$ be nonzero polynomial of degree $n \in \mathbb{N}_0$, then there exists $\mu > 0$ and $R > 1$ such that 
 $$ |f(z)| \geq \mu |z|^n $$
 for all $z \in \C$ with $|z| > R$. In particular, $|f(z)| > \mu$.
 \end{lemma}
 
 \begin{corollary}
 Fundamental Theorem of Algebra
 \end{corollary}
 
 Proof: Let $f$ be a polynomial without roots. By Lemma 2.36, there exists $R > 0$ such that $\frac{1}{f}$ is bounded on $\C - B_R(0)$. Obviously $\frac{1}{f}$ is bounded on $\overline{B_R(0)}$, and so by Liouville's Theorem, $\frac{1}{f}$ is constant.
 
 \begin{proposition}
 Let $r > 0$, $\sum \alpha_n z^n$, $\sum \beta_n z^n$ be a power series both with radius of convergence at least $r$. Then for all $n \in \mathbb{N}_0$ define 
 $$ \gamma_n = \alpha_n\beta_0 + \alpha_{n-1}\beta_1 + ,,, + \alpha_0\beta_n$$
 Then the power series $\sum \gamma_n z^n$ has radius of convergence at least $r$.
 \end{proposition}
 
 Proof: Let $f(z) = \sum \alpha_n z^n$ and $g(z) = \sum \beta_n z^n$ and $f, g: B_r(0) \to \C$ be holomorphic on its domain. Then $f g$ is also holomorphic on $B_r(0)$ and thus is analytic by Theorem 2.27; the power series of $fg$ has radius at least $r$. If $z \in B_r(0)$, then 
 $$ (fg)(z) = \sum_{n=0}^{\infty} \frac{(fg)^{(n)}(0)}{n!} z^n $$  
 Then we can verify that 
 $$ \frac{(f*g)^{(n)}(x)}{n!} = \frac{1}{n!} \sum_{k=0}^{\infty} \binom{n}{k} f^{(k)}(x)g^{(n-k)}(x) = ... = \gamma_n $$

\begin{proposition}
Let $U \subset \C$ be open, $R >0$ such that $\overline{B_R(0)} \subset U$, and $f: U \to \C$ be holomorphic. Then $\forall z \in B_R(0)$ we have that 
$$ f(z) = \frac{1}{2\pi} \int_0^{2\pi} f(Re^{i\theta}) \frac{R^2 - |z|^2}{|Re^{i\theta} - z|^2} d\theta $$
\end{proposition}

Proof: Let $z_0 \in B_R(0)$, then 
$$ z \mapsto \frac{f(z)}{R^2 - \bar{z}_0z} $$
is holomorphic on $B_{R^2 / |z_0|^2} \cap U$. Then we have that $\forall z \in B_R(0)$, 
$$ \frac{f(z)}{R^2 - \bar{z}_0z} = \frac{1}{2\pi i} \int_{\Gamma_R(0)} \frac{f(w)}{R^2 - \bar{z_0}w} \frac{1}{w-z}dw $$
Now replace $z_0$ by $z$, and $R^2$ by $w\bar{w}$ to get 
$$ \frac{f(z)}{R^2 - |z|^2} = \frac{1}{2\pi i}\int_{\Gamma_R(0)} \frac{f(w)}{w\bar{w} - \bar{z}w} \frac{1}{w-z} dw = \frac{1}{2\pi i} \int_{\Gamma_R(0)} \frac{f(w)}{|w - z|^2} \frac{1}{w} dw $$
$$ = \frac{1}{2\pi}\int_0^{2\pi} \frac{f(Re^{i\theta})}{|Re^{i\theta} - z|^2}dw $$
as desired.

\subsection{Locally Uniform Convergence}
\begin{definition}
Let $U\subset\C$ be open, $f, f_1, f_2,... : U \to \C$ be continuous. We say $\lim_{n\to\infty} f_n = f$ locally uniform on $U$ if $\forall a \in U$, there exists $r > 0$ such that $B_r(a) \subset U$ and $\lim_{n\to\infty} f_n|_{B_r(a)} = f|_{B_r(a)}$ uniformly. 
\end{definition}

\begin{example}
Place-holder
\end{example}

\begin{theorem}
Weierstrass: Let $U\subset\C$ be open, $f_1, f_2, f_3,... :U \to \C$ with $f_k$ holomorphic, and $\lim_{n\to\infty}f_n = f$ locally uniformly. Then $f$ is holomorphic and $\lim_{n\to\infty}f_n' = f'$ locally uniformly.
\end{theorem}

Proof: Let $a \in U$. So on $B_R(a)$ the convergence is uniform. Then $f$ is continuous and by Theorem 2.29, 
$$ \int_{\partial\Delta} f_k(z)dz = 0, \hspace{3mm} \forall k, \hspace{3mm} \forall \Delta \subset B_R(a) $$
Thus this implies 
$$ \int_{\partial\Delta} f(z)dz = 0, \hspace{3mm} \forall k, \hspace{3mm} \forall \Delta \subset B_R(a) $$
and so by Morera's Theorem, $f$ is holomorphic on $B_R(a)$. Now let $a \in U$, $r > 0$ and $B_{2r}(a) \subset U$ such that $\lim f_n = f$ luniformly on $\overline{B_{2r}(a)}$. Thus we have that $\lim f_n = f$ uniformly on $\partial B_r(a)$. We have that for all $z \in B_{\frac{r}{2}}(a)$,
$$ |f^{(n)}(z)| \leq \frac{n!2^{n+1}}{r^n} \max_{w \in \partial B_r(a)} |f(w)| $$
Thus we have that 
$$ |f'(z) - f'_n(z)| \leq \frac{4}{r} \max_{w\in B_r(a)} |f(w) - f_n(w)| $$
Thus $\lim f_n' = f'$ uniformly on $B_{\frac{r}{2}}(a)$.

\begin{corollary}
Let $r > 0$, $f, f_1, f_2, ...: B_r(0) \to \C$ holomorphic, and let $\sum \alpha_n^{(k)}z^n$ be the power series of $f_k$. Then if $\lim f_k = f$ locally uniformly, then $\alpha_n = \lim_{k \to \infty}\alpha_n^{(k)}$ for all $n \in \mathbb{N}$ where $\sum \alpha_n z^n$ is the power series of $f$. 
\end{corollary}

Proof: By induction on the previous theorem, $\lim_{k \to\infty} f_k^{(n)} = f^{(n)}$. In particular, the result follows at $z = 0$,
$$ f_k^{(n)}(0) = n! \alpha_n^{(k)} $$

\section{Basic Theory}

\subsection{Introduction}
\begin{proposition}
Let $U$ be open, nonempty. Then the following are equivalent 
\begin{enumerate}[(I)]
\item For any open $U_1, U_2$, with $U = U_1 \cup U_2$, $U_1 \cap U_2 = \emptyset$, then $U = U_1$ or $U = U_2$.
\item $\forall p, q\in U$, there exists piecewise smooth path $\gamma^* \subset U$ such that $\gamma(a) = p$, $\gamma(b) = q$
\item Every continuous function $f:U \to\{0,1\}$ is constant.
\end{enumerate}
\end{proposition}

Proof: Suppose (I) is true. Let $p \in U$ and define $U_1 = \{ q \in U | \exists \gamma, \gamma(a) = p, \gamma(b) = q\}$, ie the set of points which are path connected to $p$. $U$ is open, so given $q \in U_1$, there exists $r>0$ such that $B_r(q) \subset U$. We have that for all $q' \in B_r(q)$, there is a path $\gamma'$ from $q$ to $q'$ (a ball is convex). Thus $B_r(q) \subset U_1$, and so $U_1$ is open. Define $U_2 = U - U_1$. Note that $U_2$ must be open: let $u \in U_2$ and suppose there exists $r >0$ such that $B_r(u) \subset U$. If it has nontrivial intersection with $U_1$, then since $B_r(u) \subset U$ and there is a path between the intersection and $u$, we have that $u \in U_1$, a contradiction. Thus $B_r(u) \subset U_2$, and su $U_2$ is open. Thus by (I), $U = U_1$, so (II) is true.

Suppose (II) is true. Let $\varphi:U \to \{0, 1\}$ be a continuous function such that it is non-constant, there exists $p,q$ such that $\varphi(p) = 0$, $\varphi(q) = 1$. By (II) there exists a piecewise smooth curve $\gamma$ inside $U$, such that $\gamma(a) = p$, $\gamma(b) = q$. We have that $\varphi \circ \gamma: [a,b] \to \{0, 1\}$ is continuous, and so the intermediate value theorem gives us $c$ in $[a,b]$ such that $\varphi(\gamma(c)) = \frac{1}{2}$, a contradiction.

Suppose (III) is true. Let $U_1$, $U_2$ be open with $U_1 \cup U_2 = U$ and $U_1 \cap U_2 = \emptyset$. Define $\varphi:U \to \{0,1\}$ by 
$$ \varphi(p) = \begin{cases} 0 & p \in U_1 \\ 1 & p \in U_2 \end{cases} $$
We can verify that $\varphi$ is continuous because its inverse image for every open set in $\{0,1\}$ is open, and so it is constant. Thus $U = U_1$ or $U = U_2$. 

\begin{definition}
The above conditions define connectedness of an open set.
\end{definition}

\begin{definition}
We can define a domain or region as an open non-empty connected subset of $\C$.
\end{definition}

\begin{corollary}
If $U$ is open and connected, $f: U\to\C$ holomorphic with $f' = 0$, then $f$ is constant.
\end{corollary}

\subsection{Zeros of Holmorphic Functions}
\begin{definition}
Let $U$ be open, $a \in U$ with $f: U \to \C$ and $f(a) = 0$. Then we say that $a$ is an \textbf{isolated singularity} for $f$ if there exists $r>0$ such that $B_r(a)\subset U$ and $\forall z \in B_r(a) - \{ a\}$, we have $f(z) \neq 0$.

We say $a$ is a zero of infinite order if $f^{(n)}(a) = 0$ for all $n \in \mathbb{N}$.

We say $a$ is a zero of finite order (m) if there exists $m \in \mathbb{N}$ such that $f^{(m)}(a) \neq 0$, and $f^{(k)}(a) = 0$ for all $k < m$.
\end{definition}

\begin{proposition} 
Let $U$ be open, $a\in U$, and $f:U\to\C$ be holomorphic, and that $f(a) = 0$. Then there exists $r > 0$ such that $B_r(a) \subset U$ and either 
\begin{enumerate}[(I)]
\item $f(z) = 0$ for all $z \in B_r(a)$
\item $f(z) \neq 0$ for all $z \in B_r(a) - \{a\}$
\end{enumerate}
Case (I) occurs if and only if $a$ is a zero of infinite order. If $a$ is a zero of finite order, say $N \in \mathbb{N}_0$ then there exists unique $g: B_r(a) \to \C$ holomorphic, $g(a) = 0$ and $f(z) = (z-a)^ng(z)$ for all $z \in B_r(a)$.
\end{proposition}

Proof: Take $R > 0$ such that $B_R(a) \subset U$. BY Theorem 2.27, $f$ has a power series, $\sum \alpha_n (z- a)^n$ which is convergent uniformly on $B_R(a)$. Recall $\alpha_n = \frac{f^{(n)}(a)}{n!}$. If $\alpha_n = 0$ for all $n$ then $f(z) = 0$ for all $z \in B_R(a)$ and Case (I) is satisfied with $r = R$.

Suppose it is not a zero of order $\infty$. Take $N \in \mathbb{N}$ minimal such that $f^{(N)}(a) \neq 0$, so $\alpha_0 = \alpha_1 = ... \alpha_{n-1} = 0$. Then 
$$ f(z) = \sum_{n=N}^\infty \alpha_n(z-a)^n = (z-a)^N\sum_{k=0}^{\infty}\alpha_{N+k}(z-a)^k $$
for all $z \in B_R(a)$. Define $g:B_R(0)\to\C$ by $g(z) = \sum_{n=0}^{\infty} \beta_n(z-a)^n$ where $\beta_n = \alpha_{N + n}$. Restrict the domain of $g$ such that the image of $g$ does not have $0$ (possible by continuity). Then $g$ is holomorphic (because its analytic?), and satisfies the conditions of Case (II).

\begin{lemma}
Let $U$ be open, $u \in U$, and $f$ holomorphic. Let $N \in \mathbb{N}$, then $f$ has a zero at $a$ of order $N$ if and only if 
$$ \lim_{z\to a} \frac{f(z)}{(z-a)^N}$$
exists and is nonzero.
\end{lemma}

Proof: Forward direction comes from the previous proposition. Now consider 
$$ \lim_{z\to a} \frac{f(z)}{(z-a)^N} $$
where $f(a) = 0$. We know by the previous proposition that $a$ is not a zero of infinite order. Let $M$ be the order of $a$. Then there exists $g: U \to \C$ non-zero at $a$ and $f(z) = (z-a)^Mg(z)$. Therefore
$$ \lim_{z\to a} \frac{(z-a)^M g(z)}{(z-a)^N} $$
exists and is nonzero. Thus 
$$ \lim_{z \to a}(z-a)^{M-N}g(z) \neq 0 $$
and so $M = N$.

\begin{theorem}
Let $U$ be open and connected with $a \in U$, $r > 0$. Suppose $f:U \to \C$ be holomorphic and $f|_{B_r(a)} = -$. Then $f = 0$ on $U$.
\end{theorem}

Proof: Let 
$$U_1 = \{ p \in U | \exists s > 0, f|_{B_s(p)} = 0\}$$
$$ U_2 = \{ p \in U | \exists s > 0, f(z) \neq 0, z \in B_s(p) - \{ p \} \} $$
Clearly both $U_1$ and $U_2$ are open, and we have that $U = U_1 \cup U_2$ by Proposition 3.6. Thus since $U$ is connected, $U = U_1$ or $U = U_2$. Since $a \in U_1$, we have $U = U_1$, and we are done. 

\begin{corollary}
Let $U$ be open, connected, $V \subset U$ be open and non-empty. Then $f,g: U \to \C$ holomorphic and $f|_V = g|_V$ implies $f = g$.
\end{corollary}

\begin{corollary}
Let $U$ be open and connected, $f:U\to\C$. Suppose there exists a sequence of different zeroes of $f$ which converge to a point in $U$. Then $f = 0$. 
\end{corollary}
Proof: By Proposition 3.6, $f$ is zero on some open ball centered at the limit point. Thus $f$ is $0$ on $U$ by Theorem 3.8.

\subsection{Isolated Singularities}
\begin{definition}
Let $U$ be open, $a \in U$, and $f: U - \{ a\}$ be holomorphic, We say that $a$ is an \textbf{isolated singularity} of $f$. We say that $a$ is a \textbf{removable singularity} of $f$ is there exists holomorphic function $g: U \to \C$ such that $g | _{U - \{a\}} = f$. 
\end{definition}

\begin{theorem}
Riemann: Let $U$ be open, $a \in U$, and $f: U - \{ a \} \to \C$ be holomorphic. Suppose $\exists r > 0$ with $B_r(a) \subset U$ and $f|_{B_r(a) - \{a\}}$ is bounded, then $a$ is a removable singularity. 
\end{theorem}

Proof: Define the function $h: U \to \C$ by 
$$ h(z) = \begin{cases} (z - a)f(z) & z \in U - \{ a\} \\ 0 & z = a \end{cases} $$
$h$ is holomorphic on $U - \{ a \}$. Since $f$ is bounded, $\lim_{z \to a} h(z) = 0$, so $h$ is continuous on $U$. Then $h$ is holomorphic on $U$ by Lemma 2.30 (Cauchy-Goursat's Theorem and Morera's Theorem). Proposition 1.14(II) gives continuous $g: U \to \C$ such that 
$$ h(z) = h(a) + (z-a)g(z) $$
Note that $g$ is holomorphic on $g: U - \{ a \}$ but continuous at $a$, and so by Lemma 2.30 again $g$ is holomorphic. Since $h(a) = 0$, $g$ is defined as desired. 

\begin{theorem}
Casorati-Weierstrass: Let $U$ be open, $a \in U$, $f: U - \{ a \} \to \C$ be holomorphic. Then one of the following 3 occurs:
\begin{enumerate}[(I)]
\item point $a$ is removable
\item There exists $m \in \mathbb{N}$ and $c_1, ..., c_m \in \C$ such that $c_m \neq 0$ and the function 
$$ z \mapsto f(z) - \sum_{k=1}^{m}\frac{c_k}{(z-a)^k} $$
gas a removable singularity at $a$.
\item For all $r > 0$ such that $B_r(a) \subset U$, the set $f(B_r(a) - \{a\})$ is dense in $\C$ 
\end{enumerate}
\end{theorem}

Proof: Clearly only one at a time is possible. Suppose III is not the case. That means there exists $w \in \C$, $r > 0$, and $\mu > 0$ such that $B_r(a) \subset U$, and we have that $|f(z)-w| \geq \mu$ for all $z \in B_r(a) - \{ a \}$. The idea is that $\C$ contains a ball centered at $w$ that doesn't contain an element of the image of $f|_{B_r(a) - \{ a\}}$. Define $g : B_r(a) - \{ a \} \to \C$ as $z \mapsto \frac{1}{f(z) - w}$. $g$ is holomorphic on its domain and bounded, so Riemann's Theorem implies that $a$ is a removable singularity of $g$. Introduce $h: B_r(a) \to \C$ such that $h$ is equal to $g$ on $B_r(a) - \{ a \}$ and that $h$ is holomorphic. We have two cases: the first is that $h(a) \neq  0$, then $f$ is bounded in the neighborhood of $a$ and so $f$ has removable singularity at $a$ which is case I. Suppose $h(a) = 0$. Then $h$ has a zero at $a$ of finite order, so there exists $m \in \mathbb{N}$ and function $k: B_r(a) \to \C$ holomorphic (and nonzero at $a$) such that 
$$ h(z) = (z - a)^mk(z) $$
We can assume $k \neq 0$ on $B_r(a)$ (just restrict it so that it happens). Thus we have that 
$$ f(z) = w + \frac{1}{(z-a)^m} \frac{1}{k(z)} $$
for all $z \in B_r(a) - \{ a\}$. Since $\frac{1}{k(z)}$ is holomorphic, we can take the power series representation, $\sum_{n} \alpha_n(z-a)^n$ where $\alpha_0 \neq 0$, and we get that 
$$ f(z) = w + \frac{1}{(z-a)^m}\sum_n \alpha_n (z-a)^n = w + \sum_{n=0}^{\infty}\alpha_n (z-a)^{n-m} $$
as desired in case II, with $c_n = \alpha_{m-n}$.

\begin{definition}
In case III of the previous theorem, $a$ is said to be an essential singularity. 

In case II, $f$ is said to have a pole of order $m$ at $a$. If $m = 1$, then we say that the pole is simple.
\end{definition}

\begin{corollary}
Let $U$ be open, $a \in U$, with $f:U - \{ a\} \to \C$ holomorphic. Let $m \in \mathbb{N}$. Then $f$ has a pole of order $m$ at $a$ if and only if 
$$ \lim_{z \to a} (z-a)^m f(z) $$
exists and is non-zero. 
\end{corollary}

\subsection{ The Homotopy Theorem}

\begin{definition}
Let $U \subset \C$ be open, $\gamma_0:[a_0,b_0] \to \C$, $\gamma_1:[a_1,b_1] \to \C$ be two closed curves. Then $\gamma_0$ is $U$-homotopic to $\gamma_1$ if there exists continuous map $\Phi : [0,1]\times[0,1]\to U$ such that 
$$ \Phi(t, 0) = \gamma_0(a_0 + t(b_0 - a_0)) $$
$$\Phi(t, 1) = \gamma_1(a_1 + t(b_1 - a_1)) $$
$$\Phi(0, s) = \Phi(1, s), \forall s \in [0,1]$$
\end{definition}

We say that $\gamma_0$ is null homotopic if it is $U$-homotopic to a constant curve. 

Note that $U$-homotopy is an equivalence relation on curves in $U$.

\begin{lemma}
Let $K \subset \C$ be sequentially compact. 
\begin{enumerate}[(I)]
\item Let $U \subset \C$ be open and $K \subset U$, then $\exists \epsilon > 0$ such that $\forall z \in K$ and $B_{\epsilon}(z) \subset U$
\item Let $f: K \to \C$ be continuous and $\epsilon > 0$. Then there exists $\delta > 0$ such that $\forall z,w \in K$ with $|z - w| < \delta$ it follows that $|f(z) - f(w)| < \epsilon$
\end{enumerate}
\end{lemma}

Proof: (I) Suppose not. Then for all $n \in \mathbb{N}$ there exists $z_n \in K$ such that $B_{\frac{1}{n}}(z_n)$ is not a subset of $U$. $K$ is compact implies that $z_n$ has a convergent subsequence inside $K$, $z_{n_k} \to z \in K$. Because $z \in K \subset U$, there exists $N \in \mathbb{N}$ such that $B_{\frac{2}{N}}(z) \subset U$. There also exists $k \geq N$ such that 
$$ |z_{n_k} - z| < \frac{1}{N} $$
Then for all $w \in B_{\frac{1}{n_k}} (z_{n_k})$ one has 
$$ |w -z | \leq |w - z_{n_k}| + |z_{n_k} - z| < \frac{1}{n_k} + \frac{1}{N} \leq \frac{2}{N} $$ 
Thus $B_{\frac{1}{n_k}}(z_{n_k}) \subset U$, a contradiction.

\begin{theorem}
(Homotopy Theorem): Let $U$ be open, $f: U \to \C$ holomorphic. Let $\gamma_0, \gamma_1$ be two closed curves in $U$ which are $U$-homotopic. Then
$$ \int_{\gamma_0} f(z)dz = \int_{\gamma_1} f(z)dz $$
If $\gamma_0$ is null-homotopic, then $\int_{\gamma_0}f(z)dz = 0$. 
\end{theorem}

Proof: Without loss of generality let $\gamma_0, \gamma_1 : [0,1] \to \C$. Let $\Phi$ be a homotopy of $\gamma_0$ and $\gamma_1$. By Theorem 3.17, there exists $\epsilon > 0$ and $N \in \mathbb{N}$ such that 
$$ B_{\epsilon}(\Phi(t, s)) \subset U, \hspace{3mm} \forall (t,s) \in [0,1] \times [0,1] $$ 
(because continuous image of compact is compact) and we have that 
$$ |\Phi(t_1, s_1) - \Phi(t_2, s_2)| < \epsilon \hspace{3mm} \forall (t_i, s_i) \in [0,1]^2 $$
with $|(t_1, s_1) - (t_2, s_2)| \leq \frac{2}{N}$. For all $m,n \in \mathbb{Z}_N$, and let $z_{n,m} = \Phi(\frac{n}{N}, \frac{m}{N})$. Then we have that 
$$z_{n-1, m-1}, z_{n,m-1}, z_{n-1,m}, z_{n,m} \in B_{\epsilon}(z_{n,m})$$
Note the counterclockwise curve formed by the 4 points listed above is within the convex set of the ball, so by Cauchy's theorem the integral around it is $0$. Thus starting from the bottom left corner, the integral right and up is integral to the path integral up and right. (and some more stuff like adding up things)

\begin{definition}
An open set $U \subset \C$ is simply connected if every closed curve in $U$ is null-homotopic in $U$. 
\end{definition}

Remark: The ball is simply connected. Any open convex set is simply connected. $B_1(0) \cup B_1(37)$ is simply connected. 

\begin{theorem}
Let $U$ be open, simply connected, $f: U \to \C$ holomorphic. If $\gamma$ is closed, then $\int_{\gamma}f(z)dz = 0$
\end{theorem}

\begin{corollary}
Let $U$ be open, connected and simply connected, with $f: U \to \C$ holomorphic. Then there exists $F: U \to \C$ holomorphic such that $F' = f$. If $p \in U$ fixed, then $F(z) = \int_\gamma f(w)dw$ where $\gamma(0) = p$, $\gamma(1) = z$.
\end{corollary}

Proof: Fix $p$, for z choose $\gamma$ piecewise smooth from $p$ to $z$ (by connectedness). Then $F(z) = \int_{\gamma}f(z)dz$ is independent of the choice of $\gamma$ by Theorem 3.18. If $a \in U$, $r > 0$ such that $B_r(a) \subset U$. Then 
$$ F(z) = F(a) + \int_{[a,z]} f(z)dz $$
Then Corollary 2.24 gives $F$ holomorphic on $B_r(a)$. 

\begin{corollary}
Let $U$ be open, connected, and simply connected, with $f: U \to \C$ holomorphic. Suppose $f(z) \neq 0$ for all $z \in U$. Then there exists holomorphic function $g: U \to \C$ such that $e^g = f$
\end{corollary}

Proof: Consider $\frac{f'}{f}$ which is holomorphic. Then there exists $F: U \to \C$ holomorphic such that $F' = \frac{f'}{f}$. Then we have that 
$$ (fe^{-F})' = f'e^{-F} - fF'e^{-F} = 0 $$
so $fe^{-F}$ is constant since $U$ is connected. Then fix $a \in U$ and then there exists $w \in \C$ such that $e^w = f(a)$. Establish for $z \in U$ 
$$ f(z) = f(z)e^{-F(z)}e^{F(z)} = f(a)e^{-F(a)}e^{F(z)} = e^{w - F(a) + F(z)} $$
as desired.

\begin{corollary}
Let $U$ be open, $f: U \to \C$ holomorphic, $\gamma_0, \gamma_1:[0,1] \to U$ be piecewise smooth curves such that $\gamma_0(0) = \gamma_1(0)$ and $\gamma_0(1) = \gamma_1(1)$. If we have $\Phi$ continuous such that $\Phi(0, s) = \gamma_0(s)$ for all $s$ and $\Phi(1,s) = \gamma_1(s)$ for all $s$, then $\int_{\gamma_0} f(z)dz = \int_{\gamma_1} f(z)dz$.
\end{corollary}

Proof: Modify $\Phi$ to create $\tilde{\Phi}$ in the setting of Theorem 3.18. 

\begin{remark}
There does not exist function $f: \C - \{0\} \to \C$ holomorphic such that $f' = \frac{1}{z}$.
\end{remark}

Proof: Let $\gamma = \Gamma_1(0)$. Then we have that $\int_{\gamma} \frac{1}{z}dz = 2\pi i$, but if the statement was true, by Cauchy's Theorem we have $\int_{\gamma} \frac{1}{z} dz = 0$.

\subsection{The Complex Logarithm Function}

\begin{definition}
For all $x \in (0, \infty)$ we have that $\log x = \int_1^x \frac{1}{t}dt$. Thus we define 
$$ \log : \C - (-\infty, 0] \to \C, \hspace{5mm} z \mapsto \int_{[1,z]} \frac{1}{w} dw $$
\end{definition}

\begin{proposition}
\begin{enumerate}[1.]
\item $\log$ is holomorphic, $\log'(z) = \frac{1}{z} $
\item $e^{\log z} = z $
\item We have that 
$$ \log z = \log |z| + 2i \text{arctan} \frac{\Im(z)}{\Re(z) + |z|} $$
In addition, $\Im(\log z) \in (- \pi, \pi)$.
\item $\log$ is a bijection onto $\{ w \in \C : |\Im(w)| < \pi\}$
\end{enumerate}
\end{proposition}

\begin{lemma}
Let $a \in \C - (-\infty, 0]$ and $n \in \mathbb{N}$. Then $a^n = e^{n\log a}$ and $a^{-n} = e^{-n \log a}$.
\end{lemma}

\begin{definition}
Let $a \in \C - (-\infty, 0], z \in \C$ and define $a^z = e^{z \log a}$.
\end{definition}

Note that $(ab)^z \neq a^z b^z$ with this definition.

\begin{example}
Any example that messes around with the branch cut illustrates the above.
\end{example}

\begin{example}
Another example that demonstrates $(a^b)^c \neq a^{bc}$.
\end{example}

\subsection{Laurent Series}

\begin{definition} 
If $0 \leq R_1 < R_2 \leq \infty$, and $a \in \C$. Then define the \textbf{annulus} to be 
$$ A(a, R_1, R_2) = \{ z \in \C | R_1 < |z - a| < R_2 \} $$
\end{definition}

\begin{definition}
Let $z_k \in \C$, $k \in \mathbb{Z}$. We say that the series $\sum z_k$ is convergent if both the sequences $N \mapsto \sum_{k=1}^{\infty} z_k$ and $N \mapsto \sum_{k=1}^{\infty} z_{-k}$ are convergent. In that case we write 
$$ \sum_{k = - \infty}^\infty z_k = \sum_{k=0}^\infty z_k + \sum_{k=1}^{\infty} z_{-k} $$
If $a \in \C$, then a Laurent series about $a$ is a series of the form $\sum \alpha_k (z-a)^k$ with $\alpha_k \in \C$ for all $k \in \Z$. The $\alpha_k$ are called the coefficients of the Laurent series. 
\end{definition}

\begin{theorem}
Let $0 \leq R_1 < R_2 \leq \infty$, and $a \in \C$. Suppose $f: A(a, R_1, R_2) \to \C$ holomorphic, and $r \in (R_1, R_2)$ for all $k \in \Z$ define 
$$ \alpha_k = \frac{1}{2\pi i} \int_{\Gamma_r(a)} \frac{f(w)}{(w-a)^{k+1}}dw $$
Then $\alpha_k$ is independent of $r$, the series $\sum \alpha_k (z - a)^k$ is convergent and 
$$ f(z) = \sum_{k=-\infty}^\infty \alpha_k (z - a)^k $$
\end{theorem}

Proof: WLOG let $a = 0$. The independence of $\alpha_k$ is direct from the homotopy theorem. Let $z \in A(0, R_1, R_2)$ and choose $r_1$, $r_2$ such that $R_1 < r_1 < |z| < r_2 < R_2$. Define $g:A(0, R_1, R_2) \to \C$ by 
$$ g(w) = \begin{cases} \frac{f(w) - f(z)}{w- z} & w \neq z \\ f'(z) & w = z \end{cases} $$
By Riemann's Theorem (Theorem 3.12) we have that $g$ is bounded near $z$, and we get that $g$ is holomorphic. By the homotopy theorem, 
$$ \int_{\Gamma_{r_1}(0)} g(w)dw = \int_{\Gamma_{r_2}(0)} g(w)dw $$

Note that since $z$ does not lie on both of the curves, we can expand the integrand without fear to get the expression
$$ \int_{\Gamma_{r_2}(0)} \frac{f(w)}{w-z} dw - \int_{\Gamma_{r_1}(0)} \frac{f(w)}{w - z} dw = f(z) \left( \int_{\Gamma_{r_2}(0)} \frac{1}{w-z} dw - \int_{\Gamma_{r_2}(0)} \frac{1}{w-z}dw \right) $$
$$ = f(z) * 2\pi i $$

We have $r_1 < |z|$ so for $w \in \partial B_{r_1}(0)$ we define 
$$ h_n(w) = \left(\frac{w}{z}\right)^n, \hspace{3mm} \forall n \in \mathbb{N} $$ 
By Weierstrass's M-test, we have $\sum h_n$ is uniformly convergent. We also have that 
$$ -\frac{1}{2\pi i} \int_{\Gamma_{r_1}(0)} \frac{f(w)}{w-z}dw = \frac{1}{2\pi i * z} \int_{\Gamma_{r_1}(0)} \frac{f(w)}{1 - \frac{w}{z}} dw $$
$$ = \frac{1}{2\pi i} \frac{1}{z} \int_{\Gamma_{r_1}(0)} f(w) * \sum_{n=0}^{\infty} \left(\frac{w}{z}\right)^n dw  = \sum_{n=0}^\infty \frac{1}{2\pi i} \frac{1}{z} \int_{\Gamma_{r_1}(0)} f(w) \left(\frac{w}{z}\right)^n dw $$
$$ = \sum_{k=1}^\infty \alpha_{-k} z^{-k} $$
By Theorem 2.15, 
$$ \frac{1}{2\pi i} \int_{\Gamma_{r_2}(0)} \frac{f(w)}{w-z} dw = \sum_{n=0}^\infty \alpha_n z^n $$
proving the desired.

\begin{lemma}
For all $k \in \Z$, $\alpha_k \in \C$, and let $0 \leq R_1 < R_2 \leq \infty$. Suppose $\forall z \in A(a, R_1, R_2)$ the series $\sum \alpha_k (z - a)^k$ converges. Define $f: A(a, R_1, R_2) \to \C$ by $f(z) = \sum_{k=-\infty}^\infty \alpha_k(z-a)^k$. Then $f$ is holomorphic and $\forall n \in \Z$, $r \in (R_1, R_2)$ we have 
$$ \alpha_n = \frac{1}{2\pi i} \int_{\Gamma_r(a)} \frac{f(w)}{(w-z)^{n+1}} dw $$
\end{lemma}

\begin{definition}
From above $\sum \alpha_k (z-a)^k$ is the Laurent series of $f$. Define $h: A(a, R_1, R_2)$ by $h(z) = \sum_{k=-\infty}^{-1} \alpha_k (z - a)^k$ called the principal part of $f$ about $a$. In the case that $R_1 = 0$ we say that $alpha_{-1}$ is the residue of $f$ at $a$.
We write 
$$ \alpha_{-1} = \text{Res}_\alpha(f) $$
\end{definition}

\begin{proposition}
Let $f:A(a, 0, R_2) \to \C$, and $f(z) = \sum_{k=-\infty}^\infty \alpha_k(z-a)^k$. Let $m \in \mathbb{N}$. Then one has the following:
\begin{enumerate}[(a)]
\item $a$ is a removable singularity if and only if $\alpha_k = 0$ for all negative $k$.
\item $a$ is an isolated pole $\alpha_k = 0$ if and only if for all $k \in \Z$ with $k < -m$ we have that $\alpha_{-m} \neq 0$.
\item $a$ is an essential singularity of $f$ if and only if the set $\{ \alpha_k | k \in \mathbb{N} \}$ is of infinite size. 
\item If $h$ is the principal part, then $f - h$ has a removable singularity at $a$.
\end{enumerate}
\end{proposition}

\begin{lemma}
Let $a \in \C$, $r > 0$, $f: B_r(a) - \{ a \} \to \C$ holomorphic. Suppose $f$ has pole of order $m$ at $a$. Then 
$$ \text{Res}_a(f) = \frac{1}{(m-1)!} \lim_{z \to a} g^{(m-1)}(z) $$
where $g(z) = (z - a)^m f(z) $
\end{lemma}

\subsection{The Residue Theorem}
Let $U \subset \C$ be open, $a \in U$ and $f: U - \{ a \} \to \C$ holomorphic. Let $r > 0$ such that $B_r(a) \subset U$. Then $f$ has a Laurent series $\sum \alpha_k (z-a)^k$ about $a$ on the annulus $B_r(a) - \{ a \}$. Let $h$ be the principal part of $f$ about $a$. Let $\gamma$ be a piecewise smooth closed curve in $U - \{ a \}$ which is null-homotopic in $U$. Then the function $z \mapsto \sum_{k=2}^\infty \alpha_{-k}(z-a)^{-k}$ is the derivative of the holomorphic function $z\mapsto \sum_{k=2}^\infty \frac{\alpha_{-k}}{1 - k} (z-a)^{-k+1}$ on $\C - \{ a \}$. So 
$$ \int_{\gamma} \sum_{k=2}^\infty \alpha_{-k}(z-a)^{-k} dz = 0 $$
by Proposition 2.13. Hence 
$$ \frac{1}{2\pi i} \int_{\gamma} h(z)dz = \frac{1}{2\pi i} \frac{\alpha_{-1}}{z-a}dz = (\text{Res}_a f) * \Ind_\gamma(a) $$

\begin{theorem}
Let $U \subset \C$ be open, and $A \subset U$. Suppose $U - A$ is open. Let $f: U \sla A \to \C$ be a holomorphic function. Suppose for all $a \in A$ the function $f$ has an isolated singularity at $a$. Let $\gamma$ be a closed piecewise smooth curve in $U \sla A$ which is null-homotopic in $U$. Then the set 
$$ \{ a \in A : \Ind_\gamma (a) \neq 0 \} $$
is finite. Write 
$$ \{ a \in A : \Ind_\gamma (a) \neq 0 \} = \{ a_1, ..., a_N\} $$
with $N \in \mathbb{N}$ chosen minimal. Then 
$$ \frac{1}{2\pi i} \int_{\gamma} f(z)dz = \sum_{k=1}^N (\Res_{a_k} f) \Ind_\gamma(a_k) $$
\end{theorem}

Proof: Let $\Phi$ be a homotopy from $\gamma$ to a constant curve in $U$. If $a \in A \sla \Phi([0, 1] \times [0,1])$ then $\Phi$ is a homotopy from $\gamma$ to a constant curve in $\C \sla \{ a \}$. Hence $\Ind_{\gamma}(a) = \frac{1}{2\pi i} \int_\gamma \frac{1}{z-a} dz = 0$ by the homotopy theorem. So
$$\{ a \in A: \Ind_{\gamma}(a) \neq 0 \} \subset A \cap \Phi([0, 1] \times [0, 1]) $$
Note that $\Phi([0,1] \times [0,1])$ is sequentially compact. Therefiore if the set $ A \cap \Phi([0, 1] \times [0, 1])$ is infinite, there exists a convergent sequene of different elements of $A$. This sequence converges to a point $p \in \Phi([0,1] \times [0,1]) \subset U$. But there is an $r > 0$ such that $f$ is holomorphic on $B_r(p) \sla \{ p\}$, a contradiction. This establishes the first part of the theorem.

Let $V = (U \sla A) \cup \{a_1, ..., a_N\}$. Then $V$ is open and $\gamma$ is null-homotopic in $V$ via $\Phi$. For all $k \in \{ 1 ,..., N\}$ let $h_k$ be the principal part of $f$ at $b_k$. Then $f - \sum_{k=1}^N h_k$ is holomorphic on $V \sla \{a_1, ..., a_N\}$ with removable singularities at $a_1, .., a_N$ by Proposition 3.35(IV). Therefore 
$$ \int_{\gamma} (f - \sum_{k=1}^N h_k(z) dz = 0 $$ 
by the homotopy theorem. Hence 
$$ \frac{1}{2\pi i} \int_\gamma f(z)dz = \sum_{k=1}^N \frac{1}{2\pi i} \int_\gamma h_k(z)dz = \sum_{k=1}^N (\Res_{a_k} f) \Ind_\gamma(a_k) $$

\begin{lemma}
Joran's Lemma: Let $f: \{z \in \C: \Im z \geq 0 \} \to \C$ be a continuous function and suppose that 
$$ \lim_{R \to \infty} \sup \{ |f(Re^{it})| : t \in [0, \pi]\} = 0 $$
For all $R > 0$ define $\gamma_R : [0, \pi] \to \C$ by $\gamma_R(t) = Re^{it}$. Fix $m > 0$. Then 
$$ \lim_{R \to \infty} \int_{\gamma_R} e^{imz} f(z) dz = 0 $$
\end{lemma}

\begin{example}
Note that using the residue theorem we can calculate $\int_0^\infty \frac{\sin x}{x} dx$ by considering the integral of $f(z) = \frac{e^{iz}}{z}$ around the closed curve of the half circle with radius $R$ and taking the limit as $R \to \infty$. 
\end{example}

\begin{example}
We can also use the residue theorem to verify that 
$$ \sum_{n=1}^\infty \frac{1}{n^2} = \frac{\pi^2}{6} $$
\end{example}

\subsection{The Maximum Principle}
\begin{lemma} 
Let $U$ be open, $f:U\to\C$ holomorphic, $\gamma$ piecewise smooth closed curve in $U$, $b \in U \setminus \gamma*$. Then 
$$ \frac{1}{2\pi i} \int_\gamma \frac{f'(z)}{f(z) - b} dz = \Ind_{f \circ \gamma} (b) $$
\end{lemma}

Proof: We have that 
$$ \Ind_{f \circ \gamma}(b) = \frac{1}{2\pi i} \int_{f\circ \gamma} \frac{1}{z-b}dz = \frac{1}{2\pi i} \int_a^b \frac{f'(\gamma(t))\gamma'(t)}{f(\gamma(t)) - b} dt $$
$$ = \frac{1}{2\pi i} \int_\gamma \frac{f'(z)}{f(z) - b} dz $$

\begin{definition}
A simple curve $\gamma$ is a closed curve such that 
$$ \Ind_\gamma(a) \in \{0, 1\} $$
for all $a \in \C \setminus \gamma^*$
\end{definition}

\begin{theorem}
Argument Principle: Let $U$ be open, connected, $\gamma$ simple curve and null-homotopic in $U$. Define 
$$U_1 = \{ a \in \C \setminus \gamma^* | \Ind_\gamma(a) = 1\} \subset U$$
Let $A \subset \C$ be a finite set, $f:U \setminus A \to \C$ be holormophic. Suppose $f$ has a pole at all $a \in A$. Suppose $f$ has no pole/zero in $\gamma^*$. Then $f$ has finitely many zeroes in $U_1$. Let $N_f$, $P_f$ be the number of zeroes, poles respectively in $U_1$ counting multiplicty. Then 
$$ N_f - P_f = \frac{1}{2\pi i} \int_\gamma \frac{f'(z)}{f(z)} dz = \Ind_{f \circ \gamma}(0) $$
\end{theorem}

Proof: Note that the poles are isolated since $A$ is a finite set. We have that $U$ is connected, and so $U \setminus A$ is connected (isolated implies ball that doesn't contain other elements of $A$, we can then reroute the path if it goes through an element of $A$ around the center). $f$ also has finitely many zeros, otherwise we can create a sequence of unique points that has a convergent subsequence (compactify $U_1$ since it is bounded) and so we have that $f$ is identically zero. Then we have that $V = U \setminus (A \cup \{ z \in U | f(z) = 0 \})$ is open (because zeros of finite multiplicities are isolated). Thus define $g:V \to \C$ by $g(z) = \frac{f'(z)}{f(z)}$. $g$ is holomorphic. First consider zeroes. Let $a \in U$ and $m \in \mathbb{N}$, and suppose $f$ has a zero at $a$ of order $m$. So there exists $h: U \to \C$ such that $h(a) \neq 0$ and $f(z) = (z-a)^mh(z)$. Locally, 
$$ g(z) = \frac{f'(z)}{f(z)} = \frac{m(z-a)^{m-1}h(z) + (z-a)^mh'(z)}{(z-a)^mh(z)} = \frac{m}{z-a} + \frac{h'(z)}{h(z)} $$
We have that the right function is holomorphic near $a$, so $\Res_a(g) = m$. If we do a similar thing for poles, where $a$ is one of order $n$, we get that 
$$ f(z) = \frac{h(z)}{(z-a)^n} $$
and so we have that $\Res_a(g) = 0$. Thus we get that answer after applying the residue theorem. 

\begin{definition}
$A \subset \C$ is \textbf{discrete} if $\exists r > 0$ such that for all $z \in A$ we have that $B_r(z) \cap A$ is finite. Let $U$ be open, $A$ discrete in $U$. We say $f$ is meromorphic on $U$ if $\text{Dom}(f) = U \setminus A$ and $f$ is holomorphic with a pole at every $a \in A$. We then say $f$ is meromorphic on $U$ with finite singularities. 
\end{definition}

\begin{corollary} 
Let $U$ be open and connected, $\gamma$ simple and null-homotopic in $U$. Suppose $f: U \to \C$ holomorphic and take $b \in \C \setminus \gamma^*$. Let $U_1 = \{ a \in U | \Ind_\gamma(a) = 1\} \subset U$. Then $f-b$ has finitely many zeros in $U_1$, and with multiplicity equal to $\frac{1}{2\pi i} \int_\gamma \frac{f'(z)}{f(z) -b} dz$.
\end{corollary}

\begin{theorem}
Rouche: Let $U$ be open, connected, $\gamma$ is simple and null-homotopic. Set $U_1 = \{ a \in \C \setminus \gamma^* | \Ind_\gamma(a) = 1\} \subset U$. Take $f, g: U \to \C$ holomorphic and suppose on $\gamma^*$ that $|g| < |f|$. Then let $N_h$ be the number of zeros counting multiplicity of a function $h$ on $U_1$. Then we have that 
$$N_f = N_{f+g} $$
\end{theorem}

Proof: For $s \in [0,1]$, define $f_s = f + sg$. We have that $f_s$ is holomorphic. Suppose for the sake of contradiction, $f_s(z) = 0$ for all $z \in \gamma^*$. Then we have 
$$ 0 = f(z) + sg(z) \implies f(z) = -sg(z) \implies |f(z)| \leq |g(z)| $$
a contradiction. Thus $f_s$ is always non-vanishing on $\gamma^*$. Define $\varphi:[0,1] \to \Z, s\mapsto \frac{1}{2\pi i} \int_\gamma \frac{f'_s(z)}{f_s(z)} dz$. We have that $\varphi$ is continuous, and since it maps to the integers we have that it is constant. Thus $N_f = \varphi(0) = \varphi(1) = N_{f+g}$, as desired. 

\begin{theorem}
Let $U$ be open, connected and $f_1, f_2, ...:U\to \C$ holomorphic. Let $f:U\to\C$ be such that $\lim_{k\to\infty}f_k = f$ locally uniform. Suppose $f \neq 0$ and let $a\in U$, $m \in \N$. Then the following are equivalent
\begin{enumerate}[(I)]
\item $f$ has a zero at $a$ with multiplicity $m$
\item $\exists r > 0$ with $B_r(a) \subset U$ such that $\forall s \in (0, r)$ there exists $N \in \N$ such that $\forall n \geq N$ the function $f_n$ has $m$ zeros in $B_s(a)$ counted with multiplicities 
\end{enumerate}
\end{theorem}

Proof: Find $r>0$ such that $B_r(a) \subset U$, local uniform convergence is uniform convergence on $B_r(a)$, and $f(z) \neq 0$ for all $z \in B_r(a) \setminus \{ a\}$. Let $s \in (0, r)$ and $\varepsilon = \min_{z \in \partial B_s(a)} |f(z)| > 0$. There also exists $N \in \N$ such that $\forall n \geq N$, $|f_n(z) - f(z)| < \epsilon$ for all $z \in B_r(a)$. Thus we have on $\Gamma_s(a)^*$, $|f(z)| > |f_n(z) - f(z)|$, and so Rouche's Theorem gives us the forward direction.

\begin{theorem}
Hurwitz: Let $U$ be open, connected, $f_1, f_2, ...: U \to \C$ holomorphic and converge locally uniformly to $f$. If $f_k's$ don't have any zeroes, then either $f$ has no zeros, or $f = 0$. 
\end{theorem}

\begin{corollary}
Let $f_1, f_2, ... : U \to \C$ be holomorphic, injective and converge locally uniformly to $f$. Then $f$ is constant or $f$ is injective.
\end{corollary}

Proof: Let $a \in U$. Then we have that $V =.U \setminus \{ a \}$ is open and connected. Set $g_n(z) = f_n(z) - f_n(a)$. Then $g_n$ converges locally uniformly to $f(z) - f(a)$ which we define to be $g(z)$. Since $g_n$ has no zero, then $g$ doesn't have zero (or is constant). Thus $f(z) \neq f(a)$ for all $z \in U \setminus \{ a \}$.

\begin{theorem}
Let $U$ be open, $f: U \to \C$ holomorphic, and nonconstnat. Then $f(U)$ is open.

Precisely, let $a \in U$, $b = f(a)$. Suppose $f - b$ has a zero at $a$ of order $N$. Then there exists open $U_0, V_0$ such that $a \in U_0$, $b \in V_0$ with $f(U_0) = V_0$. Additionally, for all $w \in V_0 \setminus \{ b\}$ we have that $f(z) - w$ has precisely $N$ zeros on $U_0$ with multiplicity one.
\end{theorem}

Proof: $f$ is not constant so $f' \neq 0$ implying that we can't have a sequence of zeros converging in $U$ (otherwise $f$ would have a zero of infinite order). There exists $r > 0$ such that $B_{2r}(a) \subset U$ and for $z \neq a$, $z \in B_r(a)$ we have that $f(z) - b \neq 0$ and $f'(z) \neq 0$. By the argument principle with $\gamma = \Gamma_r(a)$ and $f-b$ gives that 
$$ \Ind_{f\circ\gamma}(b) = N $$
Let $V_0 = \{ w \in \C \setminus (f \circ \gamma ) | \Ind_{f\circ\gamma}(w) = N\}$. It is open because $\Ind_{f\circ\gamma}$ is continuous and takes integer values. Let $U_0 = B_r(a) \cap f^{-1}(V_0)$ which is open by continuity of $f$. We have that $f(U_0) \subset V_0$. Let $w in V_0 \setminus \{ b\}$, so $\Ind_{f\circ\gamma}(w) = N$. Argument principle implies that $f-w$ has $N$ zeros counted with multiplicities in $B_r(a)$. Let $z$ be one such zero. So $f(z) = w$, which implies $V_0 \subset f(U_0)$. Then $f'(z_0) \neq 0$ so $(f-w)'(z) \neq 0$ meaning that $f-w$ has a zero of order one at $z$.

\begin{theorem}
Maximum Modulus Principle: If $U$ is open, connected, $f:U \to \C$ holomorphic, then if $|f|$ obtains a maximum, it is constant. 
\end{theorem}

\begin{theorem}
If $|f|$ obtains its minimum, its constant or $|f| = 0$.
\end{theorem}

\begin{theorem}
If $\re f$ or $\im f$ obtains a maximum then $f$ is constant. 
\end{theorem}

\begin{corollary} 
Let $U$ be open, connected, and bounded. Suppose $f: \overline{U} \to \C$ be continuous with $f|_U$ holomorphic. Then 
$$ \sup_{z\in U} |f(z)| = \max_{z \in \overline{U}} |f(z)| = \max_{z\in\partial U} |f(z)| $$
\end{corollary}

Proof: The first equality follows form compactness. For the second, $\geq$ is obviously true. $\leq$ is true by the maximum modulus principle on $U$.

\begin{corollary} 
Let $U$ be open, $f: U \to \C$ holomorphic, $a \in U, r > 0$ and $\overline{B_r(a)} \subset U$. Let 
$$M = \max_{z \in \overline{B_r(a)}} |f(z)|$$
If $f$ is not constant, then $|f|_{B_r(a)}| < M$. 
\end{corollary}


\begin{example}
Some example that really didn't make sense in lecture
\end{example}

\begin{example}
Placeholder
\end{example}

\begin{theorem}
(Schwarz's Lemma) Let $f:B_1(0) \to B_1(0)$ be holomorphic such that $f(0) = 0$. Then $|f(z)| \leq |z|$ for all $z \in B_1(0)$ and $|f'(0)| \leq 1$. Suppose for some $a \in B_1(0) \setminus \{ 0 \}$ we have $|f(z)| = |z|$. Then $f(z) = \lambda z$ for some $|\lambda| = 1$. 
\end{theorem}

Proof: Define $g:B_1(0) \to \C$ as 
$$ g(z) = \begin{cases} \frac{f(z)}{z} & z \neq 0 \\ f'(z) & z = 0 \end{cases}$$
$g$ is holomorphic on $B_1(0) \setminus \{ 0 \}$ and continuous at $0$, so $g$ is holomorphic. We have that for all $z \in \partial B_r(0)$, where $r \in (0, 1)$ we have that $|g(z)| = \frac{|f(z)|}{|z|} \leq \frac{1}{r}$. Thus we get $|g(z)| \leq 1$ on $\partial B_1(0)$, and so by Corollary 3.55, $|g(z)| \leq 1$ on $B_1(0)$. Thus $|f(z)| \leq |z|$ on $B_1(0)$. If $|g(z)| = 1$ somewhere in $B_1(0)$, then $|g(z)|$ is constant, so $f = \lambda$ (ie $f(z).= \lambda z$).

\begin{corollary}
Let $f: B_1(0) \to B_1(0)$ be a bijection with $f(0) = 0$. Suppose $f$, $f^{-1}$ are holomorphic. Then there exists $|\lambda| = 1$ such that $f = \lambda$.
\end{corollary}

Proof: By Schwarz's Lemma, we have that $|f^{-1}(z)| \leq |z|$ so that implies $|f(z)| \geq |z|$ for all $z \in B_1(0)$. Schwarz's Lemma also implies that $|f(z)| \leq |z|$, so $|f(z)| = |z|$ giving us the desired. 

\begin{theorem}
Let $U$ be open, connected, $f:U \to \C$ injective holomorphic. Let $V = f(U)$. Then $V$ is open. Let $g: V \to U$ be the inverse of $f$. Then $g$ is holomorphic, and $f'(a) \neq 0$ for all $a \in U$.
\end{theorem}

Proof: $f$ is not constant $U$ so $V$ is open by open mapping theorem. Let $a \in U$, $ b = f(a)$. Then for all $r > 0$ we have that $f(U \cap B_r(a))$ is open, so $g$ is continuous at $b$. Suppose $f'(a) = 0$, then $f(a) - b$ has a zero of order 2 and so by the open mapping theorem there exists $r> 0$ such that $\forall w \in B_r(b) \setminus \{ b \}$ the function $f -w $ has 2 different zeroes in $U$. Since $f$ is injective this is a contradiction so $f'(a) \neq 0$. Define 
$$ H(z) = \begin{cases} \frac{z-a}{f(z) - b} & z \neq a \\ \frac{1}{f'(z)} & z = a \end{cases} $$
We have that $H$ is continuous at $a$ and $H \circ g$ is continuous at $b$. Thus 
$$ \lim_{u \to b} \frac{g(u) - g(b)}{u - b} = \lim_{u \to b} H\circ g(u) = \frac{1}{f'(a)} $$
so $g$ is differentiable at $b$ and so it is holomorphic. 

\begin{proposition}
Let $U$ be open, $f: U \to \C$ holomorphic, $a \in U$ with $f'(a) \neq 0$. Then there exists $r > 0$ such that $B_r(a) \subset U$ and $f|B_r(a)$ is injective. 
\end{proposition}

\subsection{Conformal Maps}
\begin{definition}
Let $\gamma_1, \gamma_2:[0,1] \to \C$ with $\gamma_1(0) = z = \gamma_2(0)$ and $\gamma_1'(0) \neq 0 \neq \gamma_2'(0)$. Then the angle between $\gamma_1$ and $\gamma_2$ is the unique $\theta \in (-\pi, \pi]$ such that 
$$ \frac{\gamma_1'(0)}{\gamma_2'(0)} = \left| \frac{\gamma_1'(0)}{\gamma_2'(0)}\right|e^{i\theta} $$
\end{definition}

\begin{lemma}
Let $U$ be open $f: U \to \C$ holomorphic, $\gamma_1, \gamma_2$ as above and $\gamma_1(0) = \gamma_2(0) = z \in U$. If $f'(z) \neq 0$, then the angle between $\gamma_1$ and $\gamma_2$ is the same as the angle between $f \circ \gamma_1$, $f\circ \gamma_2$.
\end{lemma}

Proof: 
$$ \frac{(f\circ\gamma_1)'(0)}{(f\circ\gamma_2)'(0)} = \frac{f'(\gamma_1(0))\gamma_1'(0)}{f'(\gamma_2(0))\gamma_2'(0)} = \frac{f'(z)}{f'(z)} \frac{\gamma_1'(0)}{\gamma_2'(0)} $$

\begin{definition}
Let $U$ be open, $f: U \to \C$. Then $f$ is called \textbf{conformal} if $f'$ never vanishes. 
\end{definition}

Note that injectivity implies conformality. 

\begin{definition}
$f$ is \textbf{biholomoprhic} if it acts between $U, V \subset \C$ open and $f, f^{-1}$ are holomorphic. 
\end{definition}

\begin{definition}
$U$ and $V$ are \textbf{conformal} if there exists $f: U \to V$ biholomorphic.
\end{definition}

\begin{theorem}
Riemann Mapping Theorem: Let $U$ nonempty be an open, connected, simply connected subset of $\C$. Then $U$ is conformal to either $B_1(0)$ or $\C$.
\end{theorem}

\begin{theorem}
Let $f$ be biholomorphic, $f:B_1(0) \to B_1(0)$. Then there exists unique $a \in B_1(0)$, $|c| = 1$ such that 
$$ f(z) = c \left(\frac{z-a}{1 - \overline{a}z}\right) $$
\end{theorem}

\end{document}